%\iffalse meta-comment
%<*driver>
           \ProvidesFile{MinionPro.dtx}        [2007/02/01 v2.1 MinionPro support for LaTeX]
%</driver>
%<style>   \NeedsTeXFormat{LaTeX2e}
%<style>   \ProvidesPackage{MinionPro}         [2007/02/01 v2.1 MinionPro support for LaTeX]
%<mtcfg>   \ProvidesFile{mt-MinionPro.cfg}     [2005/02/07 v1.0 microtype config file for MinionPro]
%<fontdef> \ProvidesFile{MinionPro-FontDef.sty}[2007/02/01 v2.2 font definition macros]
%<Uextra>  \ProvidesFile{UMinionPro-Extra.fd}  [2005/01/23 v2.0 font definitions for U/MinionPro-Extra.]
%<LGR&lf>  \ProvidesFile{LGRMinionPro-LF.fd}   [2005/01/23 v2.0 font definitions for LGR/MinionPro-LF.]
%<LGR&osf> \ProvidesFile{LGRMinionPro-OsF.fd}  [2005/01/23 v2.0 font definitions for LGR/MinionPro-OsF.]
%<LGR&tlf> \ProvidesFile{LGRMinionPro-TLF.fd}  [2005/01/23 v2.0 font definitions for LGR/MinionPro-TLF.]
%<LGR&tosf>\ProvidesFile{LGRMinionPro-TOsF.fd} [2005/01/23 v2.0 font definitions for LGR/MinionPro-TOsF.]
%<LGI&lf>  \ProvidesFile{LGIMinionPro-LF.fd}   [2005/01/23 v2.0 font definitions for LGI/MinionPro-LF.]
%<LGI&osf> \ProvidesFile{LGIMinionPro-OsF.fd}  [2005/01/23 v2.0 font definitions for LGI/MinionPro-OsF.]
%<LGI&tlf> \ProvidesFile{LGIMinionPro-TLF.fd}  [2005/01/23 v2.0 font definitions for LGI/MinionPro-TLF.]
%<LGI&tosf>\ProvidesFile{LGIMinionPro-TOsF.fd} [2005/01/23 v2.0 font definitions for LGI/MinionPro-TOsF.]
%<OML&lf>  \ProvidesFile{OMLMinionPro-LF.fd}   [2007/02/01 v2.1 font definitions for OML/MinionPro-LF.]
%<OML&osf> \ProvidesFile{OMLMinionPro-OsF.fd}  [2007/02/01 v2.1 font definitions for OML/MinionPro-OsF.]
%<OML&tlf> \ProvidesFile{OMLMinionPro-TLF.fd}  [2007/02/01 v2.1 font definitions for OML/MinionPro-TLF.]
%<OML&tosf>\ProvidesFile{OMLMinionPro-TOsF.fd} [2005/01/23 v2.0 font definitions for OML/MinionPro-TOsF.]
%<OT1&lf>  \ProvidesFile{OT1MinionPro-LF.fd}   [2005/01/23 v2.0 font definitions for OT1/MinionPro-LF.]
%<OT1&osf> \ProvidesFile{OT1MinionPro-OsF.fd}  [2005/01/23 v2.0 font definitions for OT1/MinionPro-OsF.]
%<OT1&tlf> \ProvidesFile{OT1MinionPro-TLF.fd}  [2005/01/23 v2.0 font definitions for OT1/MinionPro-TLF.]
%<OT1&tosf>\ProvidesFile{OT1MinionPro-TOsF.fd} [2005/01/23 v2.0 font definitions for OT1/MinionPro-TOsF.]
%<T1&lf>   \ProvidesFile{T1MinionPro-LF.fd}    [2005/01/23 v2.0 font definitions for T1/MinionPro-LF.]
%<T1&osf>  \ProvidesFile{T1MinionPro-OsF.fd}   [2005/01/23 v2.0 font definitions for T1/MinionPro-OsF.]
%<T1&tlf>  \ProvidesFile{T1MinionPro-TLF.fd}   [2005/01/23 v2.0 font definitions for T1/MinionPro-TLF.]
%<T1&tosf> \ProvidesFile{T1MinionPro-TOsF.fd}  [2005/01/23 v2.0 font definitions for T1/MinionPro-TOsF.]
%<LY1&lf>  \ProvidesFile{LY1MinionPro-LF.fd}   [2005/01/23 v2.0 font definitions for LY1/MinionPro-LF.]
%<LY1&osf> \ProvidesFile{LY1MinionPro-OsF.fd}  [2005/01/23 v2.0 font definitions for LY1/MinionPro-OsF.]
%<LY1&tlf> \ProvidesFile{LY1MinionPro-TLF.fd}  [2005/01/23 v2.0 font definitions for LY1/MinionPro-TLF.]
%<LY1&tosf>\ProvidesFile{LY1MinionPro-TOsF.fd} [2005/01/23 v2.0 font definitions for LY1/MinionPro-TOsF.]
%<T5&lf>   \ProvidesFile{T5MinionPro-LF.fd}    [2005/01/23 v2.0 font definitions for T5/MinionPro-LF.]
%<T5&osf>  \ProvidesFile{T5MinionPro-OsF.fd}   [2005/01/23 v2.0 font definitions for T5/MinionPro-OsF.]
%<T5&tlf>  \ProvidesFile{T5MinionPro-TLF.fd}   [2005/01/23 v2.0 font definitions for T5/MinionPro-TLF.]
%<T5&tosf> \ProvidesFile{T5MinionPro-TOsF.fd}  [2005/01/23 v2.0 font definitions for T5/MinionPro-TOsF.]
%<T2A&lf>  \ProvidesFile{T2AMinionPro-LF.fd}   [2005/01/23 v2.0 font definitions for T2A/MinionPro-LF.]
%<T2A&osf> \ProvidesFile{T2AMinionPro-OsF.fd}  [2005/01/23 v2.0 font definitions for T2A/MinionPro-OsF.]
%<T2A&tlf> \ProvidesFile{T2AMinionPro-TLF.fd}  [2005/01/23 v2.0 font definitions for T2A/MinionPro-TLF.]
%<T2A&tosf>\ProvidesFile{T2AMinionPro-TOsF.fd} [2005/01/23 v2.0 font definitions for T2A/MinionPro-TOsF.]
%<T2B&lf>  \ProvidesFile{T2BMinionPro-LF.fd}   [2005/01/23 v2.0 font definitions for T2B/MinionPro-LF.]
%<T2B&osf> \ProvidesFile{T2BMinionPro-OsF.fd}  [2005/01/23 v2.0 font definitions for T2B/MinionPro-OsF.]
%<T2B&tlf> \ProvidesFile{T2BMinionPro-TLF.fd}  [2005/01/23 v2.0 font definitions for T2B/MinionPro-TLF.]
%<T2B&tosf>\ProvidesFile{T2BMinionPro-TOsF.fd} [2005/01/23 v2.0 font definitions for T2B/MinionPro-TOsF.]
%<T2C&lf>  \ProvidesFile{T2CMinionPro-LF.fd}   [2005/01/23 v2.0 font definitions for T2C/MinionPro-LF.]
%<T2C&osf> \ProvidesFile{T2CMinionPro-OsF.fd}  [2005/01/23 v2.0 font definitions for T2C/MinionPro-OsF.]
%<T2C&tlf> \ProvidesFile{T2CMinionPro-TLF.fd}  [2005/01/23 v2.0 font definitions for T2C/MinionPro-TLF.]
%<T2C&tosf>\ProvidesFile{T2CMinionPro-TOsF.fd} [2005/01/23 v2.0 font definitions for T2C/MinionPro-TOsF.]
%<TS1&lf>  \ProvidesFile{TS1MinionPro-LF.fd}   [2005/01/23 v2.0 font definitions for TS1/MinionPro-LF.]
%<TS1&osf> \ProvidesFile{TS1MinionPro-OsF.fd}  [2005/01/23 v2.0 font definitions for TS1/MinionPro-OsF.]
%<TS1&tlf> \ProvidesFile{TS1MinionPro-TLF.fd}  [2005/01/23 v2.0 font definitions for TS1/MinionPro-TLF.]
%<TS1&tosf>\ProvidesFile{TS1MinionPro-TOsF.fd} [2005/01/23 v2.0 font definitions for TS1/MinionPro-TOsF.]
%<X2&lf>   \ProvidesFile{X2MinionPro-LF.fd}    [2005/01/23 v2.0 font definitions for X2/MinionPro-LF.]
%<X2&osf>  \ProvidesFile{X2MinionPro-OsF.fd}   [2005/01/23 v2.0 font definitions for X2/MinionPro-OsF.]
%<X2&tlf>  \ProvidesFile{X2MinionPro-TLF.fd}   [2005/01/23 v2.0 font definitions for X2/MinionPro-TLF.]
%<X2&tosf> \ProvidesFile{X2MinionPro-TOsF.fd}  [2005/01/23 v2.0 font definitions for X2/MinionPro-TOsF.]
%<OT2&lf>  \ProvidesFile{OT2MinionPro-LF.fd}   [2005/01/23 v2.0 font definitions for OT2/MinionPro-LF.]
%<OT2&osf> \ProvidesFile{OT2MinionPro-OsF.fd}  [2005/01/23 v2.0 font definitions for OT2/MinionPro-OsF.]
%<OT2&tlf> \ProvidesFile{OT2MinionPro-TLF.fd}  [2005/01/23 v2.0 font definitions for OT2/MinionPro-TLF.]
%<OT2&tosf>\ProvidesFile{OT2MinionPro-TOsF.fd} [2005/01/23 v2.0 font definitions for OT2/MinionPro-TOsF.]
%<*driver>
\documentclass{ltxdoc}
\usepackage[T1]{fontenc}
\usepackage[expansion=false]{microtype}
\usepackage[minionint]{MinionPro}
\usepackage[greek,british]{babel}
\usepackage{array,booktabs,pifont,paralist}
\IfFileExists{lmodern.sty}{%
  \renewcommand{\sfdefault}{lmss}%
  \renewcommand{\ttdefault}{lmtt}%
  \DeclareMathAlphabet\mathsf    {T1}{lmss}{m}{n}%
}{}

\newcommand*\pkg[1]{\mbox{\textsf{#1}}}
\newcommand*\file[1]{\textsf{#1}}
\newcommand*\acro[1]{\textsc{\MakeLowercase{#1}}}

\newcommand*\?{\kern0.08em}

\useshorthands{:}\defineshorthand{:}{\?\string:}
\useshorthands{;}\defineshorthand{;}{\?\string;}

\makeatletter
\newcommand*\option{\@ifstar\option@default\option@}
\newcommand*\option@default[1]{\option@{#1}*}
\newcommand*\option@[1]{\textsf{#1}}
\makeatother

\newlength\optionswidth

\newcommand\tabindent{\noindent\hspace{2em}}
\newenvironment{options}{%
  \par\penalty10000\medskip\penalty10000\tabindent
  \begin{tabular}{@{}p{\optionswidth}l@{}}%
}{%
  \end{tabular}%
  \par\medskip
}
\renewcommand\theCodelineNo{%
  \normalfont\scriptsize\figureversion{tabular}\arabic{CodelineNo}}
\renewcommand\Module[1]{%
  \mathcode`\|="2353 \mathcode`\&="2026 $\langle\mathsf{#1}\rangle$}

\setcounter{secnumdepth}{2}
\setcounter{tocdepth}{2}

\begin{document}
\DocInput{MinionPro.dtx}
\end{document}
%</driver>
%\fi
%
% \GetFileInfo{MinionPro.dtx}
%
% \title{MinionPro Support for \LaTeX}
% \author{Achim Blumensath \and Andreas B\"uhmann \and Michael Zedler}
% \date{\fileversion\ -- \filedate}
% \maketitle
%
% \begingroup
% \figureversion{tabular}%
% \renewcommand\familydefault{MinionPro-TOsF}%
% \tableofcontents
% \endgroup
%
% \section{Overview}
%
% The \pkg{MinionPro} package provides support for the MinionPro
% font family from Adobe.
% You can use these fonts in a \LaTeX\ document by adding the command
% \begin{quote}
%   |\usepackage{MinionPro}|
% \end{quote}
% to the preamble.
% This will change both the text font and the math font to MinionPro.
% If you prefer another math font (such as \pkg{eulervm}) use the option
% \option{onlytext} as explained in Section~\ref{sec:Options}.
%
% \section{Interference with other packages}
% \label{sec:Interferences}
%
% The \pkg{MinionPro} package automatically loads the following packages:
% \pkg{textcomp}, \pkg{amsmath}, and \pkg{MnSymbol}.
% If you want to pass options to these packages you can
% either put the corresponding |\usepackage| command before the
% |\usepackage{MinionPro}| or you can include the options in the
% |\documentclass| command. The \pkg{MinionPro} package is \emph{not}
% compatible with \pkg{amssymb} and \pkg{amsfonts}. Please see also the
% corresponding section in the \pkg{MnSymbol} documentation.
%
% The MinionPro package includes support files for the \pkg{microtype} package
% (version~1.8 or higher), consult the package's documentation for further details.
%
% There is also a slight incompatibility with the \pkg{dcolumn} package
% which expects all figures to have the same width. If you want to use this
% package you either have to specify the \option{mathtabular} option (this is
% the brute force solution, not recommended), or you can use the
% |\figureversion{tabular}| command to switch to tabular figures in front of
% every table (much better, but also more work). In addition, \pkg{dcolumn}
% sets figures in math mode, hence the choice of math figures
% (see Section~\ref{sec:Options}) determines if text or lining figures are used.
%
% \section{Options}\label{sec:Options}
%
% \subsubsection*{Font selection}
%
% The following options specify which version of the fonts you want to use.
% The default settings are marked with an asterisk*.
%
% \settowidth{\optionswidth}{\option{proportional}, \option{prop}}\relax
% \begin{options}
%   \option*{smallfamily} & use only regular and bold face\\
%   \option{medfamily}    & use semibold face in addition to \option{smallfamily} \\
%   \option{fullfamily}   & use medium face in addition to \option{medfamily}\\
%   \option*{noopticals}  & use only the optical size Text\\
%   \option{opticals}     & use the optical sizes Caption, Text, Subhead, and Display\\
%   \option{slides}       & use only the optical size Caption (useful for slides) \\
%   \option*{normalsize}  & adapt optical sizes to the normal font size (10\,pt, 11\,pt, 12\,pt) \\
%   \option{nonormalsize} & use static settings for the optical sizes
% \end{options}
%
% Since MinionPro comes in only four different optical sizes we use a
% variable mapping from font size to the optical size. This means that,
% both for 10\,pt and 11\,pt documents, text set in |\small| size will
% use the Caption size. Sometimes it might be desirable to turn off this
% automatism -- for instance, if you want to load the \pkg{MinionPro}
% package before the |\documentclass| command. In these cases you can
% use the \option{nonormalsize} option to do so.
%
% The package also provides a way to only change the text fonts or only the math fonts.
% \begin{options}
%   \option{onlytext} & only change the text fonts\\
%   \option{onlymath} & only change the math fonts
% \end{options}
%
% \subsubsection*{Figure selection}
%
% MinionPro offers four different figure versions. A detailed
% description is given in Section~\ref{sec:fig}.
% The default version can be selected by the following options:
%
% \begin{options}
%   \option{textosf}     & use text figures in text mode\\
%   \option{mathosf}     & use text figures in math mode\\
%   \option*{osf}        & use text figures in text and math mode\\
%   \option{textlf}      & use lining figures in text mode\\
%   \option{mathlf}      & use lining figures in math mode\\
%   \option{lf}          & use lining figures in text and math mode\\
%   \option{mathtabular} & use tabular figures in math mode\\
% \end{options}
%
% \subsubsection*{Calligraphic fonts}
%
% These options specify which font is used by the |\mathcal| command.
%
% \begin{options}
%   \option*{mnsy} & use the calligraphic font from MnSymbol: $\mathcal{ABC}$\\
%   \option{cmsy}  & take the calligraphic symbols from Computer Modern:
%                    {\fontencoding{OMS}\fontfamily{cmsy}\selectfont ABC} \\
%   \option{swash} & use the swash capitals from MinionPro: \textsw{ABC}\\
%   \option{abx}   & use the calligraphic symbols provided by mathabx:
%                    \IfFileExists{mathabx.sty}{\font\testabx=mathcm10\testabx ABCabc}{}\\
%                  & (This font contains also lowercase letters, but it is not quite finished.)
% \end{options}
%
% \subsubsection*{Blackboard bold letters}
%
% You can also select different fonts for the |\mathbb| command.
%
% \begin{options}
%   \option*{amsbb}    & use the AMS blackboard font: $\mathbb{NZQRC}$\\
%   \option{fourierbb} & use the Fourier blackboard font:
%                        \IfFileExists{fourier.sty}{\font\testfour=fourier-bb at 9.5pt\testfour NZQRC}{} \\
%   \option{lucidabb}  & use the (commercial) Lucida Math blackboard font%
%\iffalse%
%                        : {\font\testluc=hlcra at 9.2pt\testluc NZQRC}%
%\fi%
%                                                                          \\
% \end{options}
%
% \subsubsection*{Greek letters}
%
% The following options specify whether you want to use upright or italic Greek
% letters in math mode.
%
% \begin{options}
%   \option*{mixedgreek} & uppercase Greek is upright, lowercase Greek is italic\\
%   \option{italicgreek} & all Greek letters are italic\\
%   \option{frenchmath}  & all Greek letters and the uppercase Roman letters are upright
% \end{options}
%
% \noindent
% Upright and italic Greek letters are also directly accessible via the commands
% |\upgamma|, |\itgamma|, |\upGamma|, |\itGamma|, etc.
%
% \subsubsection*{Miscellaneous options}
%
% \begin{options}
%   \option{minionint}
%      & take the integral symbols from MinionPro, not from\\
%      & MnSymbol: $\displaystyle\int$ instead of $\displaystyle\varint$ \\
%   \option{openg}
%      & use $\openg$ instead of $g$ in math mode. \\
%   \option{loosequotes}
%      & The quote signs of MinionPro are set rather tight. This can\\
%      & lead to undesirable spacing for apostrophes. The \option{loosequotes} \\
%      & option slightly increases the side bearings of quotes. \\
%      & This option requires \pkg{pdflatex}~1.40 and \pkg{microtype}~2.0.\\
%      & Beware that this option prevents hyphenation of words\\
%      & containing apostrophes. Such words will require explicit\\
%      & hyphenation commands |\-|.\\
%   \option{footnotefigures}
%      & use special figures for footnote marks, i.e.,\\
%      & example\begingroup\fontfamily{MinionPro-Extra}\fontencoding{U}\selectfont
%        6,9\endgroup\ instead of example\textsuperscript{\figureversion{lf}6,9}. \\
%      & This option can only be used if the footnote marks consist \\
%      & \emph{solely} of figures.
% \end{options}
%
% \section{Figure selection}
% \label{sec:fig}
%
% MinionPro offers four different figure versions. One can choose between
% \emph{text figures} (lowercase figures) and \emph{lining figures} (uppercase figures)
% and one can choose between \emph{proportional} figures (figures with different widths)
% and \emph{tabular} figures (all figures have the same width, useful mainly for tables).
%
% \medskip
% \begin{tabular}{@{}lll@{}}
% \toprule
%                & text figures & lining figures \\
% \midrule
%   proportional & \figureversion{text,proportional}0123456789
%                & \figureversion{lining,proportional}0123456789 \\
%   tabular      & \figureversion{text,tabular}0123456789
%                & \figureversion{lining,tabular}0123456789 \\
% \bottomrule
% \end{tabular}
%
% \medskip\noindent
% The |\figureversion| command can be used to switch between
% different figure versions. Possible parameters are:
%
% \begin{options}
%   \option{text}, \option{osf}          & text figures\\
%   \option{lining}, \option{lf}         & lining figures\\
%   \option{tabular}, \option{tab}       & tabular figures\\
%   \option{proportional}, \option{prop} & proportional figures
% \end{options}
%
% Nearly no common \LaTeX\ document class supports fonts with several figure
% versions. Usually it is desirable to set most text with proportional figures
% and use tabular figures only in tables and lists.
% The following hack can be used to get tabular figures in the table of contents.
% Surround your \verb|\tableofcontents| command by the lines
% \begingroup\small
% \begin{verbatim}
% \begingroup
% \figureversion{tabular}
% \renewcommand\familydefault{MinionPro-TOsF}
% \tableofcontents
% \endgroup
% \end{verbatim}
% \endgroup
% \noindent
% The above code switches to tabular text figures. If you prefer lining figures then
% you can replace the |TOsF| by |TLF|. The same method can be applied to the
% list of figures and the list of tables.
%
% Using the \pkg{koma-script} classes, some sectioning levels are typeset by default
% with a sans-serif font in the table of contents. In order to use MinionPro with
% tabular figures instead, in addition to the above code add
%
% \begingroup\small
% \medskip\tabindent
%   |\addtokomafont|\marg{sectioning level}|{\rmfamily\figureversion{tabular}}|
% \par\endgroup
%
% \smallskip\noindent
% anywhere in your preamble. For typesetting \emph{all} headlines with MinionPro, use
%
% \begingroup\small
% \begin{verbatim}
%   \addtokomafont{sectioning}{\rmfamily\figureversion{tabular}}
% \end{verbatim}
% \endgroup
% Note that if you use one of the above hacks and your section titles contain numbers then
% these will also be set in tabular figures.
%
% There is also a proper solution which works in all cases. Unfortunately,
% it requires modifications of internal \LaTeX\ commands:
%
% \begingroup\small
% \begin{verbatim}
% \def\numberline#1{\hb@xt@\@tempdima{\figureversion{tabular}#1\hfil}}
% \def\@dottedtocline#1#2#3#4#5{%
%   \ifnum #1>\c@tocdepth \else
%     \vskip \z@ \@plus.2\p@
%     {\leftskip #2\relax \rightskip \@tocrmarg \parfillskip -\rightskip
%      \parindent #2\relax\@afterindenttrue
%      \interlinepenalty\@M
%      \leavevmode
%      \@tempdima #3\relax
%      \advance\leftskip \@tempdima \null\nobreak\hskip -\leftskip
%      {#4}\nobreak
%      \leaders\hbox{$\m@th
%         \mkern \@dotsep mu\hbox{.}\mkern \@dotsep
%         mu$}\hfill
%      \nobreak
%      \hb@xt@\@pnumwidth{\hfil\normalfont\figureversion{tab}\normalcolor #5}%
%      \par}%
%   \fi}
% \end{verbatim}
% \endgroup
% The first redefinition causes the section numbers to be set in tabular
% figures, the second one does the same for the page numbers. Note that, if you
% put these definitions directly into your document then you have to surround
% them by the commands \verb|\makeatletter| and \verb|\makeatother|.
%
% Further, note that these modifications only work for the standard document classes
% \pkg{article}, \pkg{report}, and \pkg{book}. If you use other classes such as
% \pkg{koma-script} or \pkg{memoir} then you have to copy their respective definitions and insert
% the command \verb|\figureversion{tab}| at the appropriate places.
%
% \section{Additional font shapes and symbols}
%
% In addition to the normal small caps shape |sc| there is a letterspaced
% version called~|ssc|. It is accessible via the commands |\sscshape| and
% |\textssc|. In order to use the |ssc| shape throughout your document specify
% |\renewcommand{\scdefault}{ssc}| in the preamble of your document.
%
% Swash capitals like \textsw{`Canadian Mountain Holidays'} are accessed
% via the |sw| fontshape and the commands |\swshape| and |\textsw|.
%
% \begin{options}
%   |sc|  & \textsc {This is a Sample Text}\\
%   |ssc| & \textssc{This is a Sample Text}\\
%   |sw|  & \textsw {This is a Sample Text}
% \end{options}
%
% \noindent
% The \pkg{MinionPro} package provides all symbols from the \pkg{MnSymbol} package.
% Additionally, the following math symbols are available:
%
% \medskip\tabindent
% \begingroup
% \newcommand*\sym[1]{$#1$ & \ttfamily\string#1}%
% \begin{tabular}{@{}*2{ll@{\qquad}}ll@{}}
%   \sym\digamma     & \sym\varkappa       & \sym\varbeta \\
%   \sym\backepsilon & \sym\varbackepsilon & \sym\hbar \\
%   \sym\hslash      & \sym\lambdabar      & \sym\lambdaslash \\
%   \sym\jmath       & \sym\eth            & \sym\Bbbk \\
%   \sym\slashedzero & \sym\openg \\
% \end{tabular}
% \endgroup
%
% \medskip\noindent
% Small and slanted fractions are fractions with a height matching the font's
% body size. These are useful for typesetting, e.g.,
% $\cos(\smallfrac{1}{2} x + \smallfrac{3}{2} y)$ or ``\slantfrac{1}{12}~litres of
% red wine'' and can be accessed via
%
% \medskip\tabindent
% \begin{tabular}{ll@{}@{}}
%   |\smallfrac|\marg{numerator}\marg{denominator}& \smallfrac{1}{3}\ \smallfrac{5}{17}\\
%   |\slantfrac|\marg{numerator}\marg{denominator}& \slantfrac{1}{3}\ \slantfrac{5}{17}
% \end{tabular}
%
% \medskip\noindent
% Note that \emph{only} figures can be used for \meta{numerator} and \meta{denominator}.
%
% \medskip\noindent
% Ornaments can be accessed via the \pkg{pifont} package with the command
% \begin{quote}
%   |\Pisymbol{MinionPro-Extra}|\marg{number}
% \end{quote}
% The available glyphs are listed in the table below. Version 1.000 of the
% MinionPro font provides only ornaments 100--122.
%
% \medskip\noindent
% \begingroup
% \figureversion{tabular}\centering
% \newcommand*\orn[1]{#1 & \Pisymbol{MinionPro-Extra}{#1}}\relax
% \begin{tabular}{@{}*{3}{rl@{\qquad}}rl@{}}
% \toprule
%  number & glyph & number & glyph & number & glyph & number & glyph\\
% \midrule
% \orn{100} & \orn{113} & \orn{126} & \orn{139}\\
% \orn{101} & \orn{114} & \orn{127} & \orn{140}\\
% \orn{102} & \orn{115} & \orn{128} & \orn{141}\\
% \orn{103} & \orn{116} & \orn{129} & \orn{142}\\
% \orn{104} & \orn{117} & \orn{130} & \orn{143}\\
% \orn{105} & \orn{118} & \orn{131} & \orn{144}\\
% \orn{106} & \orn{119} & \orn{132} & \orn{145}\\
% \orn{107} & \orn{120} & \orn{133} & \orn{146}\\
% \orn{108} & \orn{121} & \orn{134} & \orn{147}\\
% \orn{109} & \orn{122} & \orn{135} & \orn{148}\\
% \orn{110} & \orn{123} & \orn{136} & \orn{149}\\
% \orn{111} & \orn{124} & \orn{137} & \orn{150}\\
% \orn{112} & \orn{125} & \orn{138} & \\
% \bottomrule
% \end{tabular}\par
% \endgroup
%
% \section{Language support}
%
% The following encodings are supported:
%
% \medskip\tabindent
% \begin{tabular}{@{}ll@{}}
%   Latin    & \acro{OT1,} \acro{T1,} \acro{TS1,} \acro{LY1,} \acro{T5}\\
%   Cyrillic & \acro{T2A,} \acro{T2B,} \acro{T2C,} \acro{X2,} \acro{OT2}\\
%   Greek    & \acro{LGR} (to be used with \pkg{babel}, including \option{polutonikogreek}),\\
%            & \acro{LGI} (\pkg{Ibycus} transliteration scheme)
% \end{tabular}
%
% \medskip\noindent
% In order to typeset Greek text with the \pkg{Ibycus} transliteration scheme, specify
% \begin{quote}
%   |\usepackage[ibycus,|\meta{otherlanguages}|]{babel}|
% \end{quote}
% in the preamble and consult the documentation given in |ibycus-babel.pdf|
% on \acro{CTAN}. |\setgreekfontsize| is not supported.
%
% \section{Searching for figures or for words containing ligatures in \acro{PDF} documents}
%
% Searching for figures or for words containing ligatures in \acro{PDF} documents may not
% be possible depending on the way the \acro{PDF} file was created. The following table
% gives an overview of which glyphs may cause problems.
%
% \medskip\tabindent
% \begin{tabular}{lll}
% \toprule
%   Font Version & Program                          & Problem \\
% \midrule
%   1.000        & Ghostscript, pre-1.40 pdftex     & LF/TOsF, non-standard ligatures, swashes\\
%   1.001, 2.000 & Ghostscript, pre-1.40 pdftex     & LF/OsF/TOsF, ligatures, swashes, small caps\\
%   1.00x        & Distiller, dvipdfmx              & LF/TOsF\\
%   1.00x        & pdftex 1.40                      & ok\\
%   2.000        & Distiller, dvipdfmx, pdftex 1.40 & ok\\
% \bottomrule
% \end{tabular}
%
% \section{\acro{NFSS} classification}
%
% Parenthesised combinations are provided via substitutions.
%
% \medskip\noindent
% \begingroup
% \centering\let\>=\\%
% \begin{tabular}{@{}p{7.5em}p{7em}lp{9em}@{}}
%   \toprule
%   encoding & family & series & shape \\
%   \midrule
%   \raggedright \acro{OT1,} \acro{T1,} \acro{TS1,} \acro{LY1,} \acro{T5} &
%   MinionPro-OsF, MinionPro-LF, MinionPro-TOsF, MinionPro-TLF &
%   m, b (sb, bx), eb &
%   n, it (sl), sw\footnotemark,\>
%   sc, scit (scsl, scsw),\> ssc, sscit (sscsl, sscsw) \\
%   \addlinespace
%   \midrule
%   \addlinespace
%   \raggedright \acro{LGR,} \acro{LGI,} \acro{T2A,} \acro{T2B,} \acro{T2C,} \acro{X2,} \acro{OT2} &
%   MinionPro-OsF, MinionPro-LF, MinionPro-TOsF, MinionPro-TLF &
%   m, b (sb, bx), eb &
%   n, it (sl) \\
%   \addlinespace
%   \midrule
%   \addlinespace
%   \acro{OML} &
%   MinionPro-TOsF &
%   m, b (sb, bx), eb &
%   n, it \\
%   \addlinespace
%   \midrule
%   \addlinespace
%   \acro{U} &
%   MinionPro-Extra &
%   m, b (sb, bx), eb &
%   n, it (sl) \\
%   \bottomrule
% \end{tabular}\par
% \endgroup
% \footnotetext{via substitution in \acro{TS1} encoding}
%
% \section{Version history}
%
% Version 2.0: Initial Release on \acro{CTAN}\\
% Version 2.1:
% \begin{compactitem}
%   \item added package options \option{onlytext} and \option{onlymath}
%   \item added package option \option{loosequotes}
%   \item added package option \option{openg}
%   \item added package options \option{normalsize} and \option{nonormalsize}
%   \item fixed package option \option{frenchmath}
%   \item fixed package option \option{abx}
%   \item update to \pkg{microtype} version~1.8
%   \item fixed \verb|\t| accent
%   \item fixed \verb|\r| accent in \acro{OT1} encoding
%   \item fixed slashed zero in font version 2.000
%   \item fixed arrows in \acro{TS1} and \acro{U} encodings
%   \item fixed \acro{LGR} and \acro{LGI} encodings to use $\upvarphi$ instead of $\upphi$
%   \item fixed {\fontencoding{LGR}\selectfont\char60 R} in \acro{LGI} encoding
%   \item added punctuation support in \acro{LGI} encoding (thanks to Jens Boerstinghaus)
%   \item added symbols \verb|\hslash|, \verb|\lambdabar|, \verb|\lambdaslash|
%   \item fixed side bearings of $\sigma$ in math mode
%   \item added {\small CODINGSCHEME} statements to encoding files
%   \item fixed usage of MnSymbol's ``\textbar'' in \pkg{doc.sty}'s module prefix
% \end{compactitem}
%
% \StopEventually{}
%
% \section{The main style file}
%
% \subsection{Options}
%
%    \begin{macrocode}
%<*style>
\newif\if@Mn@Text@
\newif\if@Mn@Math@
\@Mn@Text@true
\@Mn@Math@true
\DeclareOption{onlytext}{\@Mn@Text@true\@Mn@Math@false}
\DeclareOption{onlymath}{\@Mn@Text@false\@Mn@Math@true}
%    \end{macrocode}
%
% \subsubsection{Font sets}
%
% The package \pkg{MinionPro-FontDef} adapts the font definitions to the
% requested font set (see section~\ref{sec:fontdef}). So we simply pass on the
% relevant options; only MinionPro integrals are handled here in
% \pkg{MinionPro}.
%    \begin{macrocode}
\newcommand\Mn@minionint@opticals{-NoOpticals}
\newcommand\Mn@minionint@bold{-Bold}
\DeclareOption{slides}{%
  \def\Mn@minionint@opticals{-NoOpticals}%
  \PassOptionsToPackage{slides}{MinionPro-FontDef}}
\DeclareOption{noopticals}{%
  \def\Mn@minionint@opticals{-NoOpticals}%
  \PassOptionsToPackage{noopticals}{MinionPro-FontDef}}
\DeclareOption{opticals}{%
  \def\Mn@minionint@opticals{}%
  \PassOptionsToPackage{opticals}{MinionPro-FontDef}}
\DeclareOption{smallfamily}{%
  \def\Mn@minionint@bold{-Bold}%
  \PassOptionsToPackage{smallfamily}{MinionPro-FontDef}}
\DeclareOption{medfamily}{%
  \def\Mn@minionint@bold{-Semibold}%
  \PassOptionsToPackage{medfamily}{MinionPro-FontDef}}
\DeclareOption{fullfamily}{%
  \def\Mn@minionint@bold{-Semibold}%
  \PassOptionsToPackage{fullfamily}{MinionPro-FontDef}}
\DeclareOption{normalsize}{%
  \PassOptionsToPackage{normalsize}{MinionPro-FontDef}}
\DeclareOption{nonormalsize}{%
  \PassOptionsToPackage{nonormalsize}{MinionPro-FontDef}}
%    \end{macrocode}
%
% \subsubsection{Figure style}
%
%    \begin{macrocode}
\newcommand\Mn@Text@Fig{OsF}
\newcommand\Mn@Math@Fig{OsF}
\newcommand\Mn@Text@Family{MinionPro-\Mn@Text@Fig}
\newcommand\Mn@Math@Family{MinionPro-\Mn@Math@Fig}
\newcommand\Mn@Math@TFamily{MinionPro-T\Mn@Math@Fig}
\newcommand\Mn@Math@LetterShape{it}
%    \end{macrocode}
%
%    \begin{macrocode}
\DeclareOption{textosf}{\def\Mn@Text@Fig{OsF}}
\DeclareOption{textlf} {\def\Mn@Text@Fig{LF}}
\DeclareOption{mathosf}{\def\Mn@Math@Fig{OsF}}
\DeclareOption{mathlf} {\def\Mn@Math@Fig{LF}}
\DeclareOption{osf}{\ExecuteOptions{textosf,mathosf}}
\DeclareOption{lf} {\ExecuteOptions{textlf,mathlf}}
\DeclareOption{mathtabular}{\let\Mn@Math@Family\Mn@Math@TFamily}
%    \end{macrocode}
%
% \subsubsection{Calligraphic fonts}
%
% These hooks are executed once the math versions have been set up.
%    \begin{macrocode}
\newcommand\Mn@load@cal{}
\newcommand\Mn@load@bb{}
\newcommand\Mn@load@frak{}
%    \end{macrocode}
% Most options are handled by \pkg{MnSymbol}.
%    \begin{macrocode}
\DeclareOption{mnsy}{
  \PassOptionsToPackage{mnsy}{MnSymbol}
  \def\Mn@load@cal{
    \SetMathAlphabet\mathcal{boldtabular}{OMS}{MnSymbolS}{b}{n}
  }
}
\DeclareOption{cmsy}{
  \PassOptionsToPackage{cmsy}{MnSymbol}
  \def\Mn@load@cal{
    \SetMathAlphabet\mathcal{boldtabular}{OMS}{cmsy}{b}{n}
  }
}
\DeclareOption{abx}{
  \PassOptionsToPackage{abx}{MnSymbol}
  \def\Mn@load@cal{
    \SetMathAlphabet\mathcal{boldtabular}{OT1}{mathc}{b}{n}
  }
}
\DeclareOption{swash}{
  \def\Mn@load@cal{
    \DeclareMathAlphabet\mathcal         {T1}{\Mn@Math@Family} {m} {sw}
    \SetMathAlphabet\mathcal{bold}       {T1}{\Mn@Math@Family} {eb}{sw}
    \SetMathAlphabet\mathcal{tabular}    {T1}{\Mn@Math@TFamily}{m} {sw}
    \SetMathAlphabet\mathcal{boldtabular}{T1}{\Mn@Math@TFamily}{eb}{sw}}
}
%    \end{macrocode}
%
% \subsubsection{Greek letters}
%
% |\Mn@greek@Upright|, |\Mn@greek@Mixed|, and |\Mn@greek@Italic| are defined below in
% section~\ref{sec:greek} before |\Mn@load@greek| is executed.
%    \begin{macrocode}
\newcommand\Mn@load@greek{\Mn@greek@Mixed}
\DeclareOption{frenchmath}{%
  \def\Mn@load@greek{\Mn@greek@Upright}%
  \def\Mn@Math@LetterShape{n}}
\DeclareOption{mixedgreek}{%
  \def\Mn@load@greek{\Mn@greek@Mixed}}
\DeclareOption{italicgreek}{%
  \def\Mn@load@greek{\Mn@greek@Italic}}
%    \end{macrocode}
%
% \subsubsection{Blackboard bold and fraktur fonts}
%
% We have to undefine |\mathfrak| and |\mathbb| before redefining them, because
% they might be defined in such a way that |\DeclareMathAlphabet| does not
% recognize them as math alphabets and refuses to overwrite their definitions
% (e.g., package \pkg{eufrak} uses |\newcommand{\mathfrak}{\EuFrak}|).
%    \begin{macrocode}
\newcommand\Mn@load@amsbb{
  \let\mathbb\@undefined
  \let\Bbbk\@undefined
  \DeclareMathAlphabet\mathbb{U}{msb}{m}{n}
  \newcommand\Bbbk{\mathbb{\mathchar"717C}}}
\newcommand\Mn@load@lucidabb{
  \let\mathbb\@undefined
  \let\Bbbk\@undefined
  \DeclareFontFamily{U}{hlcm}{}
  \DeclareFontShape{U}{hlcm}{m}{n}{ <->s*[0.92] hlcra }{}
  \DeclareMathAlphabet\mathbb{U}{hlcm}{m}{n}
  \newcommand\Bbbk{\mathbb{k}}}
\newcommand\Mn@load@fourierbb{
  \let\mathbb\@undefined
  \let\Bbbk\@undefined
  \DeclareFontFamily{U}{futm}{}
  \DeclareFontShape{U}{futm}{m}{n}{ <->s*[0.95] fourier-bb }{}
  \DeclareMathAlphabet\mathbb{U}{futm}{m}{n}
  \newcommand\Bbbk{\mathbb{k}}}
\DeclareOption{amsbb}    {\let\Mn@load@bb\Mn@load@amsbb}
\DeclareOption{lucidabb} {\let\Mn@load@bb\Mn@load@lucidabb}
\DeclareOption{fourierbb}{\let\Mn@load@bb\Mn@load@fourierbb}
%    \end{macrocode}
%
% \subsubsection{Integrals}
%
%    \begin{macrocode}
\newcommand\Mn@load@integrals{}
\DeclareOption{minionint}{\def\Mn@load@integrals{\Mn@Decl@Minion@Ints}}
%    \end{macrocode}
%
% \subsubsection{Miscellaneous options}
%
% Footnote figures, the g glyph in math mode, extra spacing for the apostrophe.
%    \begin{macrocode}
\DeclareOption{footnotefigures}{%
  \def\@makefnmark{%
    \begingroup
    \normalfont
    \fontfamily{MinionPro-Extra}\fontencoding{U}\selectfont
    \@thefnmark
    \endgroup}}
%
\newcommand\Mn@Define@Open@g{}
\DeclareOption{openg}{%
  \def\Mn@Define@Open@g{%
    \mathcode`g="8000%
    \DeclareMathSymbol{\Mn@g}{\mathalpha}{letters}{`g}%
    \begingroup
    \lccode`\~=`\g
    \lowercase{\gdef~{\ifnum\the\mathgroup=\m@ne \openg \else \Mn@g \fi}}%
    \endgroup
  }}
%
\newcommand\Mn@Quote@Spacing{}
\DeclareOption{loosequotes}{%
  \def\Mn@Quote@Spacing{\Mn@Quote@Spacing@Loose}}
%    \end{macrocode}
%
% \subsubsection{Defaults}
%
%    \begin{macrocode}
\ExecuteOptions{amsbb,eufrak}
\ProcessOptions\relax
%    \end{macrocode}
%
% \subsection{Font declarations}
%
%    \begin{macrocode}
\RequirePackage{MinionPro-FontDef}
\@ifpackageloaded{textcomp}{}{\RequirePackage{textcomp}}

\if@Mn@Math@
  \RequirePackage{MnSymbol}
%    \end{macrocode}
% If no fraktur font is loaded then take the Euler font.
%    \begin{macrocode}
  \@ifundefined{mathfrak}{%
    \RequirePackage{eufrak}%
    \SetMathAlphabet\EuFrak{boldtabular}{U}{euf}{b}{n}}{}
\fi
%    \end{macrocode}
% By default, we use |b| for the bold series. If |MinionPro-Semibold| is not
% available this might internally be mapped to |MinionPro-Bold| (see
% \pkg{MinionPro-FontDef}).
%    \begin{macrocode}
\if@Mn@Text@
  \edef\rmdefault{\Mn@Text@Family}
  \let\ibycusdefault\Mn@Text@Family
%    \end{macrocode}
% If a recent verion of \pkg{microtype} is loaded then we implement an option
% to increase the side bearings of all quote glyphs.
%    \begin{macrocode}
  \def\Mn@Quote@Spacing@Loose{%
    \@ifpackageloaded{microtype}{}{\RequirePackage[kerning=true]{microtype}}
    \@ifundefined{SetExtraKerning}{}{
      \let\Mn@Set@Quote@Spacing\SetExtraKerning}
%      \SetExtraKerning
%        [ unit = 1em ]
%        { encoding = {OT1,T1,LGR,U,OT2,T2A,T2B,T2C,T5,X2},
%          family   = {MinionPro-OsF,MinionPro-LF,MinionPro-TOsF,MinionPro-TLF},
%          shape    = n }
%        { \textquotedblleft = {30,30},  \textquotedblright = {30,30},
%          \textquoteleft    = {30,30},  \textquoteright    = {30,30} }}
  }
  \newcommand*\Mn@Set@Quote@Spacing[3][]{}
  \Mn@Quote@Spacing
  \Mn@Set@Quote@Spacing
    [ unit = 1em ]
    { encoding = {OT1,T1,LGR,U,OT2,T2A,T2B,T2C,T5,X2},
      family   = {MinionPro-OsF,MinionPro-LF,MinionPro-TOsF,MinionPro-TLF},
      shape    = {n,it} }
    { \textquotedblleft = {30,30},  \textquotedblright = {30,30},
      \textquoteleft    = {30,30},  \textquoteright    = {30,30} }
\fi
%    \end{macrocode}
%
% \subsubsection{Math fonts}
%
% Redefine the standard math versions |normal| and |bold|.
%    \begin{macrocode}
\if@Mn@Math@
  \DeclareSymbolFont{operators}  {T1} {\Mn@Math@Family}{m} {n}
  \DeclareSymbolFont{letters}    {OML}{MinionPro-TOsF} {m} {\Mn@Math@LetterShape}
  \SetSymbolFont{operators}{bold}{T1} {\Mn@Math@Family}{eb}{n}
  \SetSymbolFont{letters}  {bold}{OML}{MinionPro-TOsF} {eb}{\Mn@Math@LetterShape}
  \DeclareMathAlphabet\mathbf    {T1} {\Mn@Math@Family}{eb}{n}
  \DeclareMathAlphabet\mathit    {T1} {\Mn@Math@Family}{m} {it}
  \SetMathAlphabet\mathit  {bold}{T1} {\Mn@Math@Family}{eb}{it}
%    \end{macrocode}
% Extra math versions |tabular| and |boldtabular|, which use tabular figures
% instead of proportional ones. These math versions can be useful in tables
% (cf.\ section~\ref{sec:Interferences}).
%    \begin{macrocode}
  \DeclareMathVersion{tabular}
  \SetSymbolFont{operators}{tabular}    {T1} {\Mn@Math@TFamily}{m}{n}
  \SetSymbolFont{letters}  {tabular}    {OML}{MinionPro-TOsF}  {m}{\Mn@Math@LetterShape}
  \SetMathAlphabet\mathit  {tabular}    {T1} {\Mn@Math@TFamily}{m}{it}

  \DeclareMathVersion{boldtabular}
  \SetSymbolFont{operators}{boldtabular}{T1} {\Mn@Math@TFamily}{eb}{n}
  \SetSymbolFont{letters}  {boldtabular}{OML}{MinionPro-TOsF}  {eb}{\Mn@Math@LetterShape}
  \SetMathAlphabet\mathit  {boldtabular}{T1} {\Mn@Math@TFamily}{eb}{it}
%    \end{macrocode}
%
%    \begin{macrocode}
  \DeclareMathAccent{\grave}   {\mathalpha}{operators}{0}
  \DeclareMathAccent{\acute}   {\mathalpha}{operators}{1}
  \DeclareMathAccent{\hat}     {\mathalpha}{operators}{2}
  \DeclareMathAccent{\tilde}   {\mathalpha}{operators}{3}
  \DeclareMathAccent{\ddot}    {\mathalpha}{operators}{4}
  \DeclareMathAccent{\mathring}{\mathalpha}{operators}{6}
  \DeclareMathAccent{\check}   {\mathalpha}{operators}{7}
  \DeclareMathAccent{\breve}   {\mathalpha}{operators}{8}
  \DeclareMathAccent{\bar}     {\mathalpha}{operators}{9}
  \DeclareMathAccent{\dot}     {\mathalpha}{operators}{10}
%    \end{macrocode}
% Execute the hooks set up above to load the various math alphabets.
%    \begin{macrocode}
  \Mn@load@bb
  \Mn@load@frak
  \Mn@load@cal
\fi
%    \end{macrocode}
%
% \subsection{Font selection}
%
% The font selection commands such as |\figureversion|, |\textsw|, and
% |\textssc| are provided by the companion package \pkg{fontaxes}, which may be
% useful for other font families as well.
%
%    \begin{macrocode}
\RequirePackage{fontaxes}[2005/05/04]
%    \end{macrocode}
% We define an additional short hand for compatibility's sake.
%    \begin{macrocode}
\let\oldstylenums\textfigures
%    \end{macrocode}
%
% \subsection{Greek letters}
% \label{sec:greek}
%
% We provide math-mode commands for each Greek letter, both italic and upright.
% Furthermore, there are three commands to select the default version of the
% letters (all upright, all italic, or capitals upright and lowercase italic).
%
% While declaring the Greek letters we collect the uppercase and lowercase
% letters in two lists. (We distinguish them by the first letter of their
% name.) These lists are then used to select the different versions.
%    \begin{macrocode}
\if@Mn@Math@
  \newcommand\Mn@greek@list@upper{}
  \newcommand\Mn@greek@list@lower{}
  \let\Mn@greek@list@upper\@gobble
  \let\Mn@greek@list@lower\@gobble
%    \end{macrocode}
% This macro holds one of the two list names.
%    \begin{macrocode}
  \newcommand\Mn@greek@list{}
  \newcommand*\Mn@greek@letter[3]{%
    \expandafter\DeclareMathSymbol
    \expandafter{\csname it#1\endcsname}{\mathord}{letters}{#2}%
    \expandafter\DeclareMathSymbol
    \expandafter{\csname up#1\endcsname}{\mathord}{letters}{#3}%
    \edef\@tempa{`\@car#1\@nil}%
    \edef\Mn@greek@list{\expandafter\noexpand\csname
      Mn@greek@list@\ifnum\uccode\@tempa=\@tempa upper\else lower\fi\endcsname}%
    \expandafter\edef\Mn@greek@list{\Mn@greek@list,#1}%
  }
%    \end{macrocode}
% We can now declare the Greek letters (left italic, right upright).
%    \begin{macrocode}
  \Mn@greek@letter{Gamma}         {'000}{'200}
  \Mn@greek@letter{Delta}         {'001}{'201}
  \Mn@greek@letter{Theta}         {'002}{'202}
  \Mn@greek@letter{Lambda}        {'003}{'203}
  \Mn@greek@letter{Xi}            {'004}{'204}
  \Mn@greek@letter{Pi}            {'005}{'205}
  \Mn@greek@letter{Sigma}         {'006}{'206}
  \Mn@greek@letter{Upsilon}       {'007}{'207}
  \Mn@greek@letter{Phi}           {'010}{'210}
  \Mn@greek@letter{Psi}           {'011}{'211}
  \Mn@greek@letter{Omega}         {'012}{'212}
  \Mn@greek@letter{alpha}         {'013}{'213}
  \Mn@greek@letter{beta}          {'014}{'214}
  \Mn@greek@letter{gamma}         {'015}{'215}
  \Mn@greek@letter{delta}         {'016}{'216}
  \Mn@greek@letter{epsilon}       {'017}{'217}
  \Mn@greek@letter{zeta}          {'020}{'220}
  \Mn@greek@letter{eta}           {'021}{'221}
  \Mn@greek@letter{theta}         {'022}{'222}
  \Mn@greek@letter{iota}          {'023}{'223}
  \Mn@greek@letter{kappa}         {'024}{'224}
  \Mn@greek@letter{lambda}        {'025}{'225}
  \Mn@greek@letter{mu}            {'026}{'226}
  \Mn@greek@letter{nu}            {'027}{'227}
  \Mn@greek@letter{xi}            {'030}{'230}
  \Mn@greek@letter{pi}            {'031}{'231}
  \Mn@greek@letter{rho}           {'032}{'232}
  \Mn@greek@letter{sigma}         {'033}{'233}
  \Mn@greek@letter{tau}           {'034}{'234}
  \Mn@greek@letter{upsilon}       {'035}{'235}
  \Mn@greek@letter{phi}           {'036}{'236}
  \Mn@greek@letter{chi}           {'037}{'237}
  \Mn@greek@letter{psi}           {'040}{'240}
  \Mn@greek@letter{omega}         {'041}{'241}
  \Mn@greek@letter{varepsilon}    {'042}{'242}
  \Mn@greek@letter{vartheta}      {'043}{'243}
  \Mn@greek@letter{varpi}         {'044}{'244}
  \Mn@greek@letter{varrho}        {'045}{'245}
  \Mn@greek@letter{varsigma}      {'046}{'246}
  \Mn@greek@letter{varphi}        {'047}{'247}
%    \end{macrocode}
% Some of the following symbols are not really Greek letters but are treated in
% the same way.
%    \begin{macrocode}
  \Mn@greek@letter{varbeta}       {'260}{'250}
  \Mn@greek@letter{varkappa}      {'261}{'251}
  \Mn@greek@letter{backepsilon}   {'262}{'252}
  \Mn@greek@letter{varbackepsilon}{'263}{'253}
  \Mn@greek@letter{digamma}       {'264}{'254}
  \Mn@greek@letter{eth}           {'266}{'256}
%    \end{macrocode}
% Go through a list |#2| of Greek letters and |\let| them be their
% |#1|-prefixed variants.
%    \begin{macrocode}
  \newcommand*\Mn@greek@select[2]{%
    \expandafter\let\expandafter\Mn@greek@list\csname Mn@greek@list@#2\endcsname
    \@for\@tempa:=\Mn@greek@list\do{%
      \expandafter\let\csname\@tempa\expandafter\endcsname
      \csname#1\@tempa\endcsname
    }%
  }
  \newcommand*\Mn@greek@Upright{%
    \Mn@greek@select{up}{upper}%
    \Mn@greek@select{up}{lower}%
  }
  \newcommand*\Mn@greek@Italic{%
    \Mn@greek@select{it}{upper}%
    \Mn@greek@select{it}{lower}%
  }
  \newcommand*\Mn@greek@Mixed{%
    \Mn@greek@select{up}{upper}%
    \Mn@greek@select{it}{lower}%
  }
%    \end{macrocode}
% Finally initialise the Greek letters.
%    \begin{macrocode}
  \Mn@load@greek
\fi
%    \end{macrocode}
%
% \subsection{pdftex to-unicode support}
%
% Old versions of MinionPro have non-standard glyph names.
%    \begin{macrocode}
\@ifundefined{pdfglyphtounicode}{}{
  \pdfglyphtounicode{uniEFD5}{03DD}% uni03DD
  \pdfglyphtounicode{uniEFED}{02D9}% dotaccent.cap
  \pdfglyphtounicode{uniEFEE}{02D8}% breve.cap
  \pdfglyphtounicode{uniEFF1}{02DB}% ogonek.cap
  \pdfglyphtounicode{uniEFF2}{00B8}% cedilla.cap
  \pdfglyphtounicode{uniEFF3}{02DA}% ring.cap
  \pdfglyphtounicode{uniEFF5}{02DC}% tilde.cap
  \pdfglyphtounicode{uniEFF7}{02C6}% circumflex.cap
  \pdfglyphtounicode{uniF628}{2030}% perthousand.oldstyle
  \pdfglyphtounicode{uniF62C}{0028}% parenleft.denominator
  \pdfglyphtounicode{uniF62D}{0029}% parenright.denominator
  \pdfglyphtounicode{uniF631}{0028}% parenleft.numerator
  \pdfglyphtounicode{uniF632}{0029}% parenright.numerator
  \pdfglyphtounicode{uniF638}{0030}% zero.slash
  \pdfglyphtounicode{uniF639}{0030}% zero.fitted
  \pdfglyphtounicode{uniF63A}{0032}% two.fitted
  \pdfglyphtounicode{uniF63B}{0033}% three.fitted
  \pdfglyphtounicode{uniF63C}{0034}% four.fitted
  \pdfglyphtounicode{uniF63D}{0035}% five.fitted
  \pdfglyphtounicode{uniF63E}{0036}% six.fitted
  \pdfglyphtounicode{uniF63F}{0037}% seven.fitted
  \pdfglyphtounicode{uniF640}{0038}% eight.fitted
  \pdfglyphtounicode{uniF641}{0039}% nine.fitted
  \pdfglyphtounicode{uniF642}{0025}% percent.oldstyle
  \pdfglyphtounicode{uniF643}{0030}% zero.taboldstyle
  \pdfglyphtounicode{uniF644}{0031}% one.taboldstyle
  \pdfglyphtounicode{uniF645}{0032}% two.taboldstyle
  \pdfglyphtounicode{uniF646}{0033}% three.taboldstyle
  \pdfglyphtounicode{uniF647}{0034}% four.taboldstyle
  \pdfglyphtounicode{uniF648}{0035}% five.taboldstyle
  \pdfglyphtounicode{uniF649}{0036}% six.taboldstyle
  \pdfglyphtounicode{uniF64A}{0037}% seven.taboldstyle
  \pdfglyphtounicode{uniF64B}{0038}% eight.taboldstyle
  \pdfglyphtounicode{uniF64C}{0039}% nine.taboldstyle
  \pdfglyphtounicode{uniF64D}{20A1}% colonmonetary.taboldstyle
  \pdfglyphtounicode{uniF64E}{20AC}% Euro.taboldstyle
  \pdfglyphtounicode{uniF64F}{0192}% florin.taboldstyle
  \pdfglyphtounicode{uniF650}{0023}% numbersign.taboldstyle
  \pdfglyphtounicode{uniF651}{00A3}% sterling.taboldstyle
  \pdfglyphtounicode{uniF652}{00A5}% yen.taboldstyle
  \pdfglyphtounicode{uniF653}{0024}% dollar.taboldstyle
  \pdfglyphtounicode{uniF654}{00A2}% cent.taboldstyle
  \pdfglyphtounicode{uniF655}{0030}% zero.denominator
  \pdfglyphtounicode{uniF656}{0031}% one.denominator
  \pdfglyphtounicode{uniF657}{0032}% two.denominator
  \pdfglyphtounicode{uniF658}{0033}% three.denominator
  \pdfglyphtounicode{uniF659}{0034}% four.denominator
  \pdfglyphtounicode{uniF65A}{0035}% five.denominator
  \pdfglyphtounicode{uniF65B}{0036}% six.denominator
  \pdfglyphtounicode{uniF65C}{0037}% seven.denominator
  \pdfglyphtounicode{uniF65D}{0038}% eight.denominator
  \pdfglyphtounicode{uniF65E}{0039}% nine.denominator
  \pdfglyphtounicode{uniF65F}{002C}% comma.denominator
  \pdfglyphtounicode{uniF660}{002E}% period.denominator
  \pdfglyphtounicode{uniF661}{0030}% zero.numerator
  \pdfglyphtounicode{uniF662}{0031}% one.numerator
  \pdfglyphtounicode{uniF663}{0032}% two.numerator
  \pdfglyphtounicode{uniF664}{0033}% three.numerator
  \pdfglyphtounicode{uniF665}{0034}% four.numerator
  \pdfglyphtounicode{uniF666}{0035}% five.numerator
  \pdfglyphtounicode{uniF667}{0036}% six.numerator
  \pdfglyphtounicode{uniF668}{0037}% seven.numerator
  \pdfglyphtounicode{uniF669}{0038}% eight.numerator
  \pdfglyphtounicode{uniF66A}{0039}% nine.numerator
  \pdfglyphtounicode{uniF66B}{002C}% comma.numerator
  \pdfglyphtounicode{uniF66C}{002E}% period.numerator
  \pdfglyphtounicode{uniF66D}{0103}% abreve.sc
  \pdfglyphtounicode{uniF66F}{0105}% aogonek.sc
  \pdfglyphtounicode{uniF671}{0107}% cacute.sc
  \pdfglyphtounicode{uniF672}{010D}% ccaron.sc
  \pdfglyphtounicode{uniF675}{010F}% dcaron.sc
  \pdfglyphtounicode{uniF676}{0111}% dcroat.sc
  \pdfglyphtounicode{uniF678}{011B}% ecaron.sc
  \pdfglyphtounicode{uniF67B}{014B}% eng.sc
  \pdfglyphtounicode{uniF67C}{0119}% eogonek.sc
  \pdfglyphtounicode{uniF67D}{011F}% gbreve.sc
  \pdfglyphtounicode{uniF684}{0133}% ij.sc
  \pdfglyphtounicode{uniF687}{0129}% itilde.sc
  \pdfglyphtounicode{uniF68A}{013A}% lacute.sc
  \pdfglyphtounicode{uniF68B}{013E}% lcaron.sc
  \pdfglyphtounicode{uniF68E}{0144}% nacute.sc
  \pdfglyphtounicode{uniF68F}{0148}% ncaron.sc
  \pdfglyphtounicode{uniF692}{0151}% ohungarumlaut.sc
  \pdfglyphtounicode{uniF695}{0155}% racute.sc
  \pdfglyphtounicode{uniF696}{0159}% rcaron.sc
  \pdfglyphtounicode{uniF698}{015B}% sacute.sc
  \pdfglyphtounicode{uniF699}{015F}% scedilla.sc
  \pdfglyphtounicode{uniF69D}{0165}% tcaron.sc
  \pdfglyphtounicode{uniF69E}{0163}% tcommaaccent.sc
  \pdfglyphtounicode{uniF6A0}{0171}% uhungarumlaut.sc
  \pdfglyphtounicode{uniF6A3}{016F}% uring.sc
  \pdfglyphtounicode{uniF6A4}{0169}% utilde.sc
  \pdfglyphtounicode{uniF6AA}{1EF3}% ygrave.sc
  \pdfglyphtounicode{uniF6AB}{017A}% zacute.sc
  \pdfglyphtounicode{uniF6AC}{017C}% zdotaccent.sc
  \pdfglyphtounicode{uniF6DC}{0031}% one.fitted
}
%    \end{macrocode}
%
% \subsection{Superior and inferior figures}
%
% We define commands to convert numbers to numerator figures and denominator figures.
%    \begin{macrocode}
\def\@for@tok#1:=#2\do#3{%
  \expandafter\def\expandafter\@fortmp\expandafter{#2}%
  \ifx\@fortmp\@empty \else
    \expandafter\@forloop@tok#2\@nil\@nil\@@#1{#3}%
  \fi}
\def\@forloop@tok#1#2#3\@@#4#5{%
  \def#4{#1}%
  \ifx #4\@nnil \else
    #5%
    \def#4{#2}%
    \ifx #4\@nnil \else
      #5\@iforloop@tok #3\@@#4{#5}%
  \fi\fi}
\def\@iforloop@tok#1#2\@@#3#4{%
  \def#3{#1}%
  \ifx #3\@nnil
    \expandafter\@fornoop
  \else
    #4\relax\expandafter\@iforloop@tok
  \fi
  #2\@@#3{#4}}
%
\newcommand*\Mn@extra@font{%
  \fontencoding{U}\fontfamily{MinionPro-Extra}\selectfont}
\newcommand*\@numerator@fig[1]{{\Mn@extra@font\@@numerator@fig{#1}}}
\newcommand*\@denominator@fig[1]{{\Mn@extra@font\@@denominator@fig{#1}}}
\newcommand*\@superior@fig[1]{{\Mn@extra@font\@@superior@fig{#1}}}
\newcommand*\@inferior@fig[1]{{\Mn@extra@font\@@inferior@fig{#1}}}
\newcommand*\@@numerator@fig[1]{%
  \@for@tok\@nf@fig:=#1\do{%
    \ifcase\@nf@fig
       \char'00%
    \or\char'01%
    \or\char'02%
    \or\char'03%
    \or\char'04%
    \or\char'05%
    \or\char'06%
    \or\char'07%
    \or\char'10%
    \or\char'11%
    \else
      \@latex@error{invalid argument to \string\@@numerator@fig}%
    \fi
    }}
\newcommand*\@@denominator@fig[1]{%
  \@for@tok\@nf@fig:=#1\do{%
    \ifcase\@nf@fig
       \char'20%
    \or\char'21%
    \or\char'22%
    \or\char'23%
    \or\char'24%
    \or\char'25%
    \or\char'26%
    \or\char'27%
    \or\char'30%
    \or\char'31%
    \else
      \@latex@error{invalid argument to \string\@@denominator@fig}%
    \fi
    }}
\newcommand*\@@superior@fig[1]{%
  \@for@tok\@nf@fig:=#1\do{%
    \ifcase\@nf@fig
       \char'60%
    \or\char'61%
    \or\char'62%
    \or\char'63%
    \or\char'64%
    \or\char'65%
    \or\char'66%
    \or\char'67%
    \or\char'70%
    \or\char'71%
    \else
      \@latex@error{invalid argument to \string\@@superior@fig}%
    \fi
    }}
\newcommand*\@@inferior@fig[1]{%
  \@for@tok\@nf@fig:=#1\do{%
    \ifcase\@nf@fig
       \char'100%
    \or\char'101%
    \or\char'102%
    \or\char'103%
    \or\char'104%
    \or\char'105%
    \or\char'106%
    \or\char'107%
    \or\char'110%
    \or\char'111%
    \else
      \@latex@error{invalid argument to \string\@@inferior@fig}%
    \fi
    }}
%    \end{macrocode}
% |\ensure@text| switches to text mode, if necessary.
%    \begin{macrocode}
\newcommand*\ensure@text[1]{%
  \ifmmode
    \Mn@Text@With@MathVersion{#1}%
  \else
    #1%
  \fi}
%    \end{macrocode}
% |\smallfrac| and |\slantfrac| assemble numerical fractions.
%    \begin{macrocode}
\newcommand*\@smallfrac[2]{%
  \leavevmode
  \setbox\@tempboxa
    \vbox{%
      \baselineskip\z@skip%
      \lineskip.25ex%
      \lineskiplimit-\maxdimen
      \ialign{\hfil##\hfil\crcr
              \vbox to 2.13ex{\vss\hbox{\@numerator@fig{#1}}\vskip.68ex}\crcr
              \leavevmode\leaders\hrule height 1.1ex depth -1.01ex\hfill\crcr
              \vtop to 1ex{\vbox{}\hbox{\@denominator@fig{#2}}\vss}\crcr
              \noalign{\vskip-1.47ex}}}%
  \dp\@tempboxa=0.49ex%
  \box\@tempboxa}
\newcommand*\@slantfrac[2]{%
  {\Mn@extra@font\@@numerator@fig{#1}\kern-0.05em/\kern-0.06em\@@denominator@fig{#2}}}
\DeclareRobustCommand*\smallfrac[2]{\ensure@text{\kern0.06em\@smallfrac{#1}{#2}\kern0.09em}}
\DeclareRobustCommand*\slantfrac[2]{\ensure@text{\kern0.06em\@slantfrac{#1}{#2}\kern0.09em}}
%    \end{macrocode}
%
% \subsection{Additional symbols}
%
% Some symbols missing from MnSymbol can be taken from MinionPro.
%
%    \begin{macrocode}
\if@Mn@Math@
  \let\hbar\undefined
  \DeclareMathSymbol{\hbar}             {\mathord}{letters}{'265}
  \DeclareMathSymbol{\uphbar}           {\mathord}{letters}{'255}
  \DeclareMathSymbol{\partial}          {\mathord}{letters}{'100}
  \DeclareMathSymbol{\uppartial}        {\mathord}{letters}{'300}
  \DeclareMathSymbol{\ell}              {\mathord}{letters}{'140}
  \DeclareMathSymbol{\upell}            {\mathord}{letters}{'340}
  \DeclareMathSymbol{\slashedzero}      {\mathord}{letters}{'257}
  \DeclareMathSymbol{\upimath}          {\mathord}{letters}{'373}
  \DeclareMathSymbol{\upjmath}          {\mathord}{letters}{'374}
  \DeclareMathSymbol{\varsmallint}      {\mathord}{letters}{'376}
  \DeclareMathSymbol{\openg}            {\mathalpha}{letters}{'267}
  \DeclareRobustCommand\lambdabar       {\middlebar\lambda}
  \DeclareRobustCommand\lambdaslash     {\middleslash\lambda}
\fi
%    \end{macrocode}
% Archaic Greek letters not provided by MinionPro.
%    \begin{macrocode}
\if@Mn@Text@
  %\def\Qoppa{\reflectbox{P}}
  %\def\Sampi{\begingroup\fontfamily{cmr}\fontencoding{LGR}\selectfont\char23\endgroup}
  \let\Stigma\stigma

  % fix \r A
  \DeclareTextCompositeCommand{\r}{OT1}{A}
     {\leavevmode\setbox\z@\hbox{!}\dimen@\ht\z@\advance\dimen@-1ex%
     \ooalign{\hss\raise.67\dimen@\hbox{\char23}\hss\crcr A}}

  \DeclareEncodingSubset{TS1}{MinionPro-LF}  {1}%
  \DeclareEncodingSubset{TS1}{MinionPro-TLF} {1}%
  \DeclareEncodingSubset{TS1}{MinionPro-OsF} {1}%
  \DeclareEncodingSubset{TS1}{MinionPro-TOsF}{1}%
  \AtBeginDocument{
    \UndeclareTextCommand{\textvisiblespace}{T1}%
    \UndeclareTextCommand{\textcompwordmark}{T1}%
    \UndeclareTextCommand{\textsterling}{T1}%
    \UndeclareTextCommand{\j}{T1}%
    \UndeclareTextCommand{\j}{LY1}%
  }
\fi
%    \end{macrocode}
%
% \subsection{Integral symbols}
%
% We can also replace the integral signs from MnSymbol by those of MinionPro.
% The following definitions provide this as an option.
%    \begin{macrocode}
\if@Mn@Math@
  \newcommand\Mn@Decl@Minion@Ints{%
%    \end{macrocode}
% Replace |MnSymbolF| by |MnSymbolFI|.
%    \begin{macrocode}
    \DeclareFontFamily{U}{MnSymbolFI}{}
    \DeclareFontShape{U}{MnSymbolFI}{m}{it}{
        <-6>  MnSymbolFI\Mn@minionint@opticals5
       <6-7>  MnSymbolFI\Mn@minionint@opticals6
       <7-8>  MnSymbolFI\Mn@minionint@opticals7
       <8-9>  MnSymbolFI\Mn@minionint@opticals8
       <9-10> MnSymbolFI\Mn@minionint@opticals9
      <10-12> MnSymbolFI\Mn@minionint@opticals10
      <12->   MnSymbolFI\Mn@minionint@opticals12
    }{}
    \DeclareFontShape{U}{MnSymbolFI}{b}{it}{
        <-6>  MnSymbolFI\Mn@minionint@bold\Mn@minionint@opticals5
       <6-7>  MnSymbolFI\Mn@minionint@bold\Mn@minionint@opticals6
       <7-8>  MnSymbolFI\Mn@minionint@bold\Mn@minionint@opticals7
       <8-9>  MnSymbolFI\Mn@minionint@bold\Mn@minionint@opticals8
       <9-10> MnSymbolFI\Mn@minionint@bold\Mn@minionint@opticals9
      <10-12> MnSymbolFI\Mn@minionint@bold\Mn@minionint@opticals10
      <12->   MnSymbolFI\Mn@minionint@bold\Mn@minionint@opticals12
    }{}
%    \end{macrocode}
%
%    \begin{macrocode}
    \DeclareSymbolFont{symbols}  {U}{MnSymbolFI}{m}{it}
    \SetSymbolFont{symbols}{bold}{U}{MnSymbolFI}{b}{it}
%    \end{macrocode}
% Make the original integral symbols available as |\var|\dots.
%    \begin{macrocode}
    \let\varint\tint
    \let\variint\tiint
    \let\variiint\tiiint
    \let\variiiint\tiiiint
    \let\varidotsint\tidotsint
    \let\varlandupint\tlandupint
    \let\varlanddownint\tlanddownint
    \let\varstrokedint\tstrokedint
    \let\varoint\toint
    \let\varoiint\toiint
    \let\varrcirclerightint\trcirclerightint
    \let\varlcirclerightint\tlcirclerightint
    \let\varrcircleleftint\trcircleleftint
    \let\varlcircleleftint\tlcircleleftint
    \let\varsumint\tsumint
%    \end{macrocode}
% Replace the symbols with the new integrals.
%    \begin{macrocode}
    \DeclareMathSymbol\tint            \mathop{symbols}{112}
    \DeclareMathSymbol\tiint           \mathop{symbols}{114}
    \DeclareMathSymbol\tiiint          \mathop{symbols}{116}
    \DeclareMathSymbol\tiiiint         \mathop{symbols}{118}
    \DeclareMathSymbol\tidotsint       \mathop{symbols}{120}
    \DeclareMathSymbol\tlandupint      \mathop{symbols}{122}
    \DeclareMathSymbol\tlanddownint    \mathop{symbols}{124}
    \DeclareMathSymbol\tstrokedint     \mathop{symbols}{126}
    \DeclareMathSymbol\toint           \mathop{symbols}{128}
    \DeclareMathSymbol\toiint          \mathop{symbols}{130}
    \DeclareMathSymbol\trcirclerightint\mathop{symbols}{132}
    \DeclareMathSymbol\tlcirclerightint\mathop{symbols}{134}
    \DeclareMathSymbol\trcircleleftint \mathop{symbols}{136}
    \DeclareMathSymbol\tlcircleleftint \mathop{symbols}{138}
    \DeclareMathSymbol\tsumint         \mathop{symbols}{140}
    \let\intop\tint
    \let\ointop\toint
  }
%    \end{macrocode}
%
%    \begin{macrocode}
  \Mn@load@integrals
\fi
%    \end{macrocode}
%
% \subsection{Open G support}
%
% We can replace the closed~$g$ with the open variant~$\openg$.
% The following definitions provide this as an option.
%    \begin{macrocode}
\if@Mn@Math@
  \Mn@Define@Open@g
\fi
%    \end{macrocode}
%
% \subsection{Logos}
%
% Correct logos.
%    \begin{macrocode}
\if@Mn@Text@
  \def\TeX{T\kern-.1667em\lower.4ex\hbox{E}\kern-.125emX\@}
  \DeclareRobustCommand{\LaTeX}{L\kern-.32em%
          {\sbox\z@ T%
           \vbox to\ht\z@{\hbox{\check@mathfonts
                                \fontsize\sf@size\z@
                                \math@fontsfalse\selectfont
                                A}%
                          \vss}%
          }%
          \kern-.15em%
          \TeX}
\fi
%    \end{macrocode}
%
% \subsection{AMS}
%
% Fix a bug in \pkg{amsmath.sty} which does not support
% math fonts without a skew char.
%    \begin{macrocode}
\def\macc@set@skewchar#1{%
  \begingroup
  \ifnum\mathgroup=\m@ne \let\@tempa\@ne
  \else
    \ifnum\skewchar\textfont\mathgroup=\m@ne \let\@tempa\@ne
    \else \let\@tempa\mathgroup
    \fi
  \fi
  \count@=\skewchar\textfont\@tempa
  \ifnum\count@=\m@ne
    \endgroup
    \def\macc@skewchar{}
  \else
    \advance\count@"7100
    \edef\@tempa{\endgroup
      \mathchardef\noexpand\macc@skewchar=\number\count@\relax}%
    \@tempa
  \fi
  #1%
}
%    \end{macrocode}
% Make the changes take effect. This concludes the main style file.
%    \begin{macrocode}
\if@Mn@Text@
  \normalfont
\fi
%</style>
%    \end{macrocode}
%
% \section{Support for character protrusion}
%
% The \pkg{microtype} configuration. All four MinionPro families use the same
% file (cf.\ section~\ref{sec:fontdef}).
%    \begin{macrocode}
%<*mtcfg>
\SetProtrusion
  [ name     = MinionPro-OT1-Roman ]
  { encoding = OT1,
    family   = {MinionPro-OsF,MinionPro-LF,MinionPro-TOsF,MinionPro-TLF},
    shape    = n }
  {
      A = {40,40},
      F = {  ,60},
      J = {90,  },
      K = {  ,50},
      L = {  ,60},
      T = {50,50},
      V = {40,40},
      W = {30,30},
      X = {50,50},
      Y = {50,50},
      k = {  ,60},
      r = {  ,80},
      t = {  ,100},
      v = {70,70},
      w = {40,40},
      x = {60,60},
      y = {70,70},
      ! = {70,180},
      ( = {60,30},    ) = {30,60},
      [ = {100,160},  ] = {160,100},
    {,} = {440,700},
      . = {660,700},
      : = {400,480},
      ; = {350,440},
      - = {700,700},
    \textendash       = {390,480},  \textemdash        = {220,270},
    \textquotedblleft = {380,250},  \textquotedblright = {250,380},
    \textquoteleft    = {670,450},  \textquoteright    = {450,670},
  }
%    \end{macrocode}
%
%    \begin{macrocode}
\SetProtrusion
  [ name     = MinionPro-T1-Roman,
    load     = MinionPro-OT1-Roman ]
  { encoding = T1,
    family   = {MinionPro-OsF,MinionPro-LF,MinionPro-TOsF,MinionPro-TLF},
    shape    = n }
  {
    023 = {  ,40}, % fft ligature
    032 = {  ,50}, % ft ligature
    191 = {30,30}, % Th ligature
    127 = {620,700}, % hyphen
    \AE = {40,  }, % AE
    \quotesinglbase = {670,670},  \quotedblbase   = {370,370},
    \guilsinglleft  = {500,360},  \guilsinglright = {360,500},
    \guillemotleft  = {320,230},  \guillemotright = {230,320},
  }
%    \end{macrocode}
%
%    \begin{macrocode}
\SetProtrusion
  [ name     = MinionPro-OT1-Italic]
  { encoding = OT1,
    family   = {MinionPro-OsF,MinionPro-LF,MinionPro-TOsF,MinionPro-TLF},
    shape    = {it,sl,sw} }
  {
      A = {120,50},
      B = {90,-50},
      C = {50,-60},
      D = {70,-30},
      E = {90,-50},
      F = {100,-40},
      G = {50,-60},
      H = {70,-40},
      I = {150,-90},
      J = {250,-130},
      K = {80,-50},
      L = {90,60},
      M = {60,-40},
      N = {70,-40},
      O = {70,-30},
      P = {70,-110},
      Q = {40,-40},
      R = {80,-50},
      S = {70,-70},
      T = {130,  },
      U = {70,-40},
      V = {120,30},
      W = {90,20},
      X = {50,  },
      Y = {160,  },
      Z = {50,-50},
      d = {60,-60},
      f = {  ,-190},
    027 = {  ,-70}, % ff ligature
      g = {-70,-70},
      i = {  ,-110},
    025 = {  ,-60}, % dotlessi
    028 = {  ,-60}, % fi ligature
    030 = {  ,-30}, % ffi ligature
      j = {-90,-150},
      p = {-40,  },
      r = {  ,80},
      t = {  ,100},
      v = {90,  },
      w = {60,10},
      x = {90,  },
      ! = {190,40},
      ( = {90,  },    ) = {90,  },
      [ = {90,90},    ] = {120,60},
    {,} = {210,680},
      . = {640,680},
      : = {380,430},
      ; = {   ,430},
      - = {750,750},
    \textquoteleft    = {690,140},  \textquoteright    = {470,230},
    \textendash       = {400,500},  \textemdash        = {220,280},
    \textquotedblleft = {520,130},  \textquotedblright = {520,130},
  }
%    \end{macrocode}
%
%    \begin{macrocode}
\SetProtrusion
  [ name     = MinionPro-T1-Italic,
    load     = MinionPro-OT1-Italic ]
  { encoding = T1,
    family   = {MinionPro-OsF,MinionPro-LF,MinionPro-TOsF,MinionPro-TLF},
    shape    = {it,sl,sw} }
  {
    023 = {  ,40}, % fft ligature
    032 = {  ,50}, % ft ligature
    191 = {80,30}, % Th ligature
    127 = {660,750}, % hyphen
    \AE = {90,-40}, % AE
    131 = {80,-30}, % Dcaron
    132 = {70,-40}, % Ecaron
    156 = {80,-60}, % IJ
    \OE = {50,-30}, % OE
    188 = {  ,-80}, % ij
    184 = {70,70}, % ydieresis
    253 = {70,70}, % yacute
    \quotesinglbase = {220,700},  \quotedblbase   = {130,400},
    \guilsinglleft  = {500,180},  \guilsinglright = {350,350},
    \guillemotleft  = {310,110},  \guillemotright = {230,230},
  }
%    \end{macrocode}
%
% We have no protruding values for small caps yet. The following stubs are
% unnecessary at the moment, but they are here as a reminder.
%
%    \begin{macrocode}
\SetProtrusion
  [ name     = MinionPro-OT1-Smallcaps ]
  { encoding = OT1,
    family   = {MinionPro-OsF,MinionPro-LF,MinionPro-TOsF,MinionPro-TLF},
    shape    = {sc,ssc} }
  {}
%    \end{macrocode}
%
%    \begin{macrocode}
\SetProtrusion
  [ name     = MinionPro-T1-Smallcaps,
    load     = MinionPro-OT1-Smallcaps ]
  { encoding = T1,
    family   = {MinionPro-OsF,MinionPro-LF,MinionPro-TOsF,MinionPro-TLF},
    shape    = {sc,ssc} }
  {}
%    \end{macrocode}
%
%    \begin{macrocode}
\SetProtrusion
  [ name     = MinionPro-OT1-SmallcapsItalic ]
  { encoding = OT1,
    family   = {MinionPro-OsF,MinionPro-LF,MinionPro-TOsF,MinionPro-TLF},
    shape    = {scit,sscit} }
  {}
%    \end{macrocode}
%
%    \begin{macrocode}
\SetProtrusion
  [ name     = MinionPro-T1-SmallcapsItalic,
    load     = MinionPro-OT1-SmallcapsItalic ]
  { encoding = T1,
    family   = {MinionPro-OsF,MinionPro-LF,MinionPro-TOsF,MinionPro-TLF},
    shape    = {scit,sscit} }
  {}
\SetProtrusion
  [ name     = MinionPro-other-Roman ]
  { encoding = {LGR,U,OT2,T2A,T2B,T2C,T5,X2},
    family   = {MinionPro-OsF,MinionPro-LF,MinionPro-TOsF,MinionPro-TLF},
    shape    = n }
  {
      ! = {70,180},
      ( = {60,30},    ) = {30,60},
      [ = {100,160},  ] = {160,100},
    {,} = {440,700},
      . = {660,700},
      : = {400,480},
      ; = {350,440},
      - = {700,700},
    \textendash       = {390,480},  \textemdash        = {220,270},
    \textquotedblleft = {380,250},  \textquotedblright = {250,380},
    \textquoteleft    = {670,450},  \textquoteright    = {450,670},
  }
\SetProtrusion
  [ name     = MinionPro-other-Italic ]
  { encoding = {LGR,U,OT2,T2A,T2B,T2C,T5,X2},
    family   = {MinionPro-OsF,MinionPro-LF,MinionPro-TOsF,MinionPro-TLF},
    shape    = {it,sl,sw} }
  {
      ! = {190,40},
      ( = {90,  },    ) = {90,  },
      [ = {90,90},    ] = {120,60},
    {,} = {210,680},
      . = {640,680},
      : = {380,430},
      ; = {   ,430},
      - = {750,750},
    \textquoteleft    = {690,140},  \textquoteright    = {470,230},
    \textendash       = {400,500},  \textemdash        = {220,280},
    \textquotedblleft = {520,130},  \textquotedblright = {520,130},
  }
%</mtcfg>
%    \end{macrocode}
%
% \section{Font definition files}
% \label{sec:fontdef}
%
% As all the font definitions look the same we introduce macros to ease the
% configuration. These macros are stored in the file
% \file{MinionPro-FontDef.sty} which is included by every \acro{FD} file.
% Note that \file{MinionPro-FontDef.sty} will be included several times and
% that we do not know in which context the code is executed. Therefore, we have
% to define all non-private commands as globals.
%
% \makeatletter
% \newcommand*\exampleconfig[2]{%
%   \begin{quote}\ttfamily
%   \expandafter\string\csname Mn@config@#1@#2\endcsname\par
%   \expandafter\expandafter\expandafter\strip@prefix
%   \expandafter\meaning\csname Mn@config@#1@#2\endcsname\par
%   \edef\@tempa{\Mn@UseConfig{#1}{#2}}%
%   \expandafter\strip@prefix\meaning\@tempa
%   \end{quote}%
% }%
% \makeatother
%
% Since this package should be loadable in an \acro{FD} file we have to
% avoid all |\preambleonly| commands.
% Therefore, we use |\ProvidesFile| instead of |\ProvidesPackage|.
%
% We add a guard so that this file is executed only once even if it is included
% multiple times.
%    \begin{macrocode}
%<*fontdef>
\ifx\Mn@DeclareFontShape\@undefined\else\endinput\fi
%    \end{macrocode}
% We distinguish between being loaded directly or via |\usepackage| in the
% preamble by checking |\@nodocument|.
%    \begin{macrocode}
\ifx\@nodocument\relax
  % \iffalse
%<*driver>
\ProvidesFile{otfontdef.dtx}
%</driver>
%<otfd>\@ifundefined{@nodocument}
%<otfd>  {\ProvidesFile{otfontdef.sty}}
%<otfd>  {\NeedsTeXFormat{LaTeX2e}\ProvidesPackage{otfontdef}}
%<*otfd|driver>
  [2005/05/05 v0.1c OpenType font definition]
%</otfd|driver>
%<*driver>
\documentclass{ltxdoc}
\newcommand\pkg[1]{\textsf{#1}}
\newcommand\acro[1]{#1}
\RecordChanges
\begin{document}
\DocInput{otfontdef.dtx}
\end{document}
%</driver>
%<*otfd>
% \fi
% 
% \GetFileInfo{otfontdef.dtx}
% 
% \title{OpenType Font Definition}
% \author{Andreas B\"uhmann}
% \date{\fileversion\ -- \filedate}
% \maketitle
% 
% \begin{abstract}
%   The package \pkg{otfontdef} provides a simple but powerful key-value-based interface for
%   font definition files. It enables you to vary multiple font
%   attributes according to the font size separately from each other.
% \end{abstract}
%
% \section{Motivation}
%
% Writing font definition files for professional OpenType fonts is tedious.
% You have many different aspects like optical size, weight, and letterspacing
% at your disposal that you might want to vary according to the font size but
% each one independently of the other. To achieve this with ordinary means
% involves intersecting size intervals, copying unchanged parts, etc., which
% leads to a lot of repetition and, in turn, to decreased maintainability.
% 
% As an example, imagine that you are about to map the font
% |T1/MinionPro-OsF/m/it| to external fonts in a font definition file. This is
% easy if there is exactly one external font:
% \begin{verbatim}
% \DeclareFontShape{T1}{MinionPro-OsF}{m}{it}{
%       <->     MinionPro-It-osf-t1
% }{}
% \end{verbatim}
% But let us assume that you have access to multiple optical sizes of that
% font and that you would like to use them (for simplicity, let us use only
% two optical sizes, Caption and Text):
% \begin{verbatim}
%       <-8.5>  MinionPro-ItCapt-osf-t1
%    <8.5->     MinionPro-It-osf-t1
% \end{verbatim}
% You then decide that it would be nice to increase the weights at small
% sizes; for instance, use Medium for sizes less than 8\,pt and Semibold for
% sizes less than 6\,pt. This implies having to split the first interval
% twice:
% \begin{verbatim}
%       <-6>    MinionPro-SemiboldItCapt-osf-t1
%      <6-8>    MinionPro-MediumItCapt-osf-t1
%      <8-8.5>  MinionPro-ItCapt-osf-t1
% \end{verbatim}
% If you finally come up with the idea of using a letterspaced version of your
% font at small sizes (less than 7\,pt), you will arrive at this
% configuration, where each of your design decisions is scattered over up to
% five lines:
% \begin{verbatim}
% \DeclareFontShape{T1}{MinionPro-OsF}{m}{it}{
%       <-6>    MinionPro-SemiboldItCapt-osf-l1-t1
%      <6-7>    MinionPro-MediumItCapt-osf-l1-t1
%      <6-8>    MinionPro-MediumItCapt-osf-t1
%      <8-8.5>  MinionPro-ItCapt-osf-t1
%    <8.5->     MinionPro-It-osf-t1
% }{}
% \end{verbatim}
% This package enables you to specify these decisions directly.
% \begin{verbatim}
% \DeclareFontShape{T1}{MinionPro-OsF}{m}{it}{
%       <->     otf* MinionPro-It
%       <-8.5>  otf* [optical=Capt]
%       <-6>    otf* [weight=Semibold]
%      <6-8>    otf* [weight=Medium]
%       <-7>    otf* [spacing=l1]
% }{}
% \end{verbatim}
% The package will then analyze your specifications, combine them, and
% generate the correct external font names according to some naming scheme.
%
% \section{Usage}
% 
% \dots
%
% \StopEventually{}
%
% \section{Implementation}
%
% \begin{macro}{\otf@disable@preamblecmds}
%   In a second we need to temporarily disable all commands that can be used
%   in the preamble only (and that occur in the current version of keyval).
%    \begin{macrocode}
\newcommand\otf@disable@preamblecmds{%
  \def\@gobble@optional{%
    \@ifnextchar[\@gobble@optional@{}%]
  }%
  \def\@gobble@optional@[##1]{}%
  \def\NeedsTeXFormat##1{\@gobble@optional}%
  \def\ProvidesPackage##1{\@gobble@optional}%
  \let\DeclareOption\@gobbletwo
  \let\ExecuteOptions\@gobble
  \def\ProcessOptions{\@ifstar\@gobble{}}%
}
%    \end{macrocode}
% \end{macro}
% Try hard to also work when loaded from inside an \acro{FD} file. We use
% |\space| because literal spaces are ignored.
%    \begin{macrocode}
\ifx\@nodocument\relax
  \PackageWarningNoLine{otfontdef}{Please\space load\space me\space
  in\space the\space preamble.\MessageBreak
  I'm\space doing\space my\space best\space to\space continue\space anyway}%
  \@ifundefined{define@key}{
    \begingroup
%    \end{macrocode}
% \pkg{keyval} uses the space token to define its commands. We make sure that
% it is the right one. (Space has catcode 9 (ignore) in \acro{FD} files.)
%    \begin{macrocode}
    \catcode32=10
    \otf@disable@preamblecmds
%    \end{macrocode}
% I have always waited to find a use for |\globaldefs|; here it is. It is
% needed for the definitions in \pkg{keyval} to survive this group and
% especially the group around the \acro{FD} file. We must be extremely careful
% not to execute definitions that we do not want to be global: Hence, we cannot
% let |\ProvidesPackage| to |\ProvidesFile| because it changes catcodes. Let us
% hope that \pkg{keyval} does not change.
%    \begin{macrocode}
    \globaldefs=1
    \input keyval.sty
    \endgroup
  }{}%
\else
  \RequirePackage{keyval}[1999/03/16 v1.13]
\fi
%    \end{macrocode}
%
% \begin{macro}{\otf@makeglobal}
%   We have to make definitions global to allow this package to be used from
%   inside \acro{FD} files,
%    \begin{macrocode}
\newcommand\otf@makeglobal[1]{
  \global\expandafter\let\csname #1\expandafter\endcsname
  \csname #1\endcsname
}
%    \end{macrocode}
% but we need not when loaded as a normal package.
%    \begin{macrocode}
\ifx\@nodocument\relax\else
  \let\otf@makeglobal\@gobble
\fi
%    \end{macrocode}
% \end{macro}
% \begin{macro}{\otf@info}
% We only show information on chosen configurations when asked to do so.
% \changes{v0.2}{2005/04/19}{Introduce trace option}
%    \begin{macrocode}
\newif\ifotf@trace\otf@tracefalse
\newcommand*\otf@info[1]{}
\ifx\@nodocument\relax\else
  \DeclareOption{trace}{\otf@tracetrue}
  \ProcessOptions\relax
  \ifotf@trace
    \def\otf@info#1{\PackageInfo{otfontdef}{#1}}
  \fi
\fi
\otf@makeglobal{otf@info}
%    \end{macrocode}
% \end{macro}
% \begin{macro}{\ifotf@options}
%   Whether we are looking for options or not when scanning the configuration.
%    \begin{macrocode}
\newif\ifotf@options
\otf@optionsfalse
\otf@makeglobal{ifotf@options}
\otf@makeglobal{otf@optionstrue}
\otf@makeglobal{otf@optionsfalse}
%    \end{macrocode}
% \end{macro}
% 
% \begin{macro}{\otf@keys}
% \begin{macro}{\otf@definekey}
%   Only the first occurrence of a key sets the associated value; other
%   occurrences are ignored.
%    \begin{macrocode}
\newcommand\otf@keys{}
\newcommand\otf@definekey[1]{%
  \define@key{otf}{#1}{%
    \@ifundefined{otf@@#1}{%
      \@namedef{otf@@#1}{##1}%
    }{}%
  }%
  \expandafter\let\csname otf@@#1\endcsname\relax
  \g@addto@macro\otf@keys{\do{#1}}%
  \otf@makeglobal{otf@@#1}%
  \otf@makeglobal{KV@otf@#1}%
}
%    \end{macrocode}
% \end{macro}
% \end{macro}
% 
% \begin{macro}{\otf@default@keys}
% \begin{macro}{\otf@definedefault}
%    \begin{macrocode}
\newcommand\otf@default@keys{}
\newcommand\otf@definedefault[2]{%
  \@namedef{KV@otf@#1@default}{#2}%
  \edef\otf@default@keys{\otf@default@keys,#1}%
  \otf@makeglobal{KV@otf@#1@default}
}
%    \end{macrocode}
% \end{macro}
% \end{macro}
% Define all font attributes
%    \begin{macrocode}
\otf@definekey{family}
\otf@definekey{weight}
\otf@definekey{shape}
\otf@definekey{optical}
\otf@definekey{variant}
\otf@definekey{figures}
\otf@definekey{spacing}
\otf@definekey{encoding}
\otf@definekey{size}
\otf@makeglobal{otf@keys}
%    \end{macrocode}
% and how their defaults are determined (if at all).
%    \begin{macrocode}
\otf@definedefault{weight}
  {\expandafter\KV@otf@weight\expandafter{\otf@Regular}}
\otf@definedefault{optical}
  {\expandafter\KV@otf@optical\expandafter{\otf@Text}}
\otf@definedefault{variant}
  {\expandafter\otf@splitname@int\f@family--\@empty}
\otf@definedefault{figures}
  {\expandafter\otf@splitname@int\f@family--\@empty}
\otf@definedefault{encoding}
  {\expandafter\KV@otf@encoding\expandafter{\f@encoding}}
\otf@definedefault{size}
  {\expandafter\KV@otf@size\expandafter{\f@size}}
\otf@definedefault{shape}
  {\expandafter\KV@otf@shape\expandafter{\otf@Regular}}
\otf@makeglobal{otf@default@keys}
%    \end{macrocode}
% 
% \begin{macro}{\otf@Regular}
% \begin{macro}{\otf@Text} 
%   All characters of these strings must have catcode 12 (other). They will be
%   matched against substrings of |\DeclareFontShape| configurations.
%    \begin{macrocode}
\newcommand*\otf@Regular{Regular}
\newcommand*\otf@Text{Text}
\newcommand*\otf@Ornaments{Ornaments}
\@onelevel@sanitize\otf@Regular
\@onelevel@sanitize\otf@Text
\@onelevel@sanitize\otf@Ornaments
\otf@makeglobal{otf@Regular}
\otf@makeglobal{otf@Text}
\otf@makeglobal{otf@Ornaments}
%    \end{macrocode}
% \end{macro}
% \end{macro}
% We register a new size function |otf|, which can then be used in \acro{FD}
% files. We procede in two phases when building the configuration for a
% requested font: In the first phase, we ignore all |otf| entries that give
% only options (no |\mandatory@arg|); from the first entry that has a
% mandatory argument we start into the second phase, where we rescan the
% configuration looking for options.
%    \begin{macrocode}
\ifx\@nodocument\relax
  \begingroup
  \def\DeclareSizeFunction#1#2{\endgroup\global\@namedef{s@fct@#1}{#2}}%
  \expandafter
\fi
\DeclareSizeFunction{otf}{%
  \ifotf@options
    \otf@get@options
  \else
    \ifx\mandatory@arg\@empty\else
      \otf@get@external@font
    \fi
  \fi
}
%    \end{macrocode}
%
% \begin{macro}{\otf@get@options}
%   Simply process all key-value pairs given in the optional argument.
%    \begin{macrocode}
\newcommand\otf@get@options{%
  \@expandtwoargs\setkeys{otf}{\optional@arg}%
}
\otf@makeglobal{otf@get@options}
%    \end{macrocode}
% \end{macro}
%
% \begin{macro}{\otf@splitname@ext}
% \begin{macro}{\otf@splitname@int}
% \changes{v0.2}{2005/05/05}{Remove erroneous space token}
%    \begin{macrocode}
\newcommand\otf@splitname@ext{}
\def\otf@splitname@ext#1-#2-#3\@empty{%
  \def\otf@@family{#1}%
  \def\otf@@shape{#2}%
  \ifx\otf@@shape\@empty
    \let\otf@@shape\relax
  \fi
}
\newcommand\otf@splitname@int{}
\def\otf@splitname@int#1-#2-#3\@empty{%
  \KV@otf@family{#1}%
  \def\@tempa{#2}%
  \ifx\@tempa\otf@Ornaments
    \KV@otf@variant{orn}%
  \else
    \KV@otf@figures{#2}%
  \fi
}
\otf@makeglobal{otf@splitname@ext}
\otf@makeglobal{otf@splitname@int}
%    \end{macrocode}
% \end{macro}
% \end{macro}
%
% \begin{macro}{\otf@get@external@font}
%   This is the master macro that coordinates the processing. We first
%   determine family and (possibly) shape from the mandatory argument.
%    \begin{macrocode}
\newcommand\otf@get@external@font{%
  \expandafter\otf@splitname@ext\mandatory@arg--\@empty
%    \end{macrocode}
% We can then rescan the current configuration for further options.
%    \begin{macrocode}
  \otf@optionstrue
  \try@size@range
%    \end{macrocode}
% If any attribute has not received a value, we use its default.
%    \begin{macrocode}
  \@expandtwoargs\setkeys{otf}{\otf@default@keys}%
%    \end{macrocode}
% After having informed the user about the values we have collected, we build
% the external font name by applying the appropriate naming scheme.
%    \begin{macrocode}
  \begingroup
  \def\do##1{\otf@showoption{##1}\MessageBreak}%
  \otf@info{Using\space configuration\MessageBreak
    \otf@keys for\space font\space\font@name}%
  \endgroup
  \@ifundefined{otf@scheme@@\otf@@family}{\otf@scheme@default}{%
    \@nameuse{otf@scheme@@\otf@@family}%
  }%
  \otf@info{Trying\space to\space load\space external\space font\MessageBreak
    `\external@font'}%
}
\otf@makeglobal{otf@get@external@font}
%    \end{macrocode}
% \end{macro}
% 
% \begin{macro}{\otf@scheme@default}
%   This is the font naming scheme used in the MinionPro project.
%   \changes{v0.2}{2005/04/20}{Improve grouping}
%    \begin{macrocode}
\newcommand\otf@head{}
\newcommand\otf@tail{}
\newcommand\otf@scheme@default{%
  \begingroup
  \edef\@tempa{\lowercase{\def\noexpand\otf@tail{%
    \otf@opt\otf@@figures
    \otf@opt\otf@@variant
    \otf@opt\otf@@spacing
    \otf@@encoding
  }}}\@tempa
  \edef\otf@head{%
    \ifx\otf@@weight\otf@Regular\else\otf@@weight\fi
    \otf@format@shape\otf@@shape
    \ifx\otf@@optical\otf@Text\else\otf@@optical\fi}%
  \ifx\otf@head\@empty
    \edef\otf@head{\otf@@family-\otf@Regular}%
  \else
    \edef\otf@head{\otf@@family-\otf@head}%
  \fi
  \edef\@tempa{\endgroup\def\noexpand\external@font{%
    \otf@head-\otf@tail\space at\space\otf@@size pt}}%
  \@tempa
}
\otf@makeglobal{otf@scheme@default}
\newcommand\otf@format@shape[1]{%
  \@ifundefined{otf@format@shape@#1}{}{\@nameuse{otf@format@shape@#1}}%
}
\newcommand\otf@format@shape@Regular{}%
\newcommand\otf@format@shape@Italic{It}%
\newcommand\otf@format@shape@It{It}%
\otf@makeglobal{otf@format@shape}
\otf@makeglobal{otf@format@shape@Regular}
\otf@makeglobal{otf@format@shape@Italic}
\otf@makeglobal{otf@format@shape@It}
%    \end{macrocode}
% \end{macro}
% \begin{macro}{\DeclareFontNamingScheme}
%   We do not make this declaration command global. The who wants to use it
%   should really load the package in preamble.
%   \changes{v0.2}{2005/04/20}{Use the term `scheme' instead of `pattern'}
%    \begin{macrocode}
\newcommand*\DeclareFontNamingScheme[1]{%
  \@namedef{otf@scheme@@#1}%
}
%    \end{macrocode}
% \end{macro}
%
% \begin{macro}{\otf@opt}
%   An optional name component, separated from the following ones by a hyphen.
%    \begin{macrocode}
\newcommand*\otf@opt[1]{%
  \ifx\relax#1\@empty\else\if\@empty#1\@empty\else#1-\fi\fi
}
%    \end{macrocode}
% \end{macro}
% \begin{macro}{\otf@showoption}
%   Format an option and its current value for displaying it.
%    \begin{macrocode}
\newcommand*\otf@showoption[1]{%
  \@spaces #1\space=\space\@ifundefined{otf@@#1}{<undefined>}{%
    \expandafter\expandafter\expandafter\strip@prefix
    \expandafter\meaning\csname otf@@#1\endcsname}
}
\otf@makeglobal{otf@opt}
\otf@makeglobal{otf@showoption}
%    \end{macrocode}
% \end{macro}
%
% \PrintChanges
% \Finale
%
% \iffalse
%</otfd>
% \fi
\endinput

\else
  \NeedsTeXFormat{LaTeX2e}
  \RequirePackage{otfontdef}
\fi
%    \end{macrocode}
% Reset |\escapechar| (which is set to $-1$ in \acro{FD} files) to make
% |\newcommand| work. The additional group does not harm; we have to
% make the important commands global anyway.
%    \begin{macrocode}
\ifx\@nodocument\relax
  \begingroup\escapechar`\\
\fi
%    \end{macrocode}
% These are the default values if it is impossible to process options.
%    \begin{macrocode}
\newcommand\Mn@option@opticals{noopticals}
\newcommand\Mn@option@fontset{smallfamily}
\newdimen\Mn@option@normalsize
\global\Mn@option@normalsize10pt
%    \end{macrocode}
% Whether we should adapt the configuration to the |\normalsize| of the
% document. This switch is only needed locally.
%    \begin{macrocode}
\newif\ifMn@option@normalsize
\Mn@option@normalsizetrue
%    \end{macrocode}
%
%    \begin{macrocode}
\ifx\@nodocument\relax\else
  \DeclareOption{slides}     {\let\Mn@option@opticals\CurrentOption}
  \DeclareOption{opticals}   {\let\Mn@option@opticals\CurrentOption}
  \DeclareOption{noopticals} {\let\Mn@option@opticals\CurrentOption}
  \DeclareOption{smallfamily}{\let\Mn@option@fontset\CurrentOption}
  \DeclareOption{medfamily}  {\let\Mn@option@fontset\CurrentOption}
  \DeclareOption{fullfamily} {\let\Mn@option@fontset\CurrentOption}
  \DeclareOption{normalsize}  {\Mn@option@normalsizetrue}
  \DeclareOption{nonormalsize}{\Mn@option@normalsizefalse}
  \ExecuteOptions{smallfamily,noopticals,normalsize}
  \ProcessOptions\relax
\fi
%    \end{macrocode}
% The method to determine the main font size is inspired by \pkg{microtype}'s
% implementation.
%    \begin{macrocode}
\ifMn@option@normalsize
  \begingroup
  \def\set@fontsize#1#2#3#4\@nil{%
    \@defaultunits\global\Mn@option@normalsize#2pt\relax\@nnil}%
  \normalsize\@nil
  \endgroup
\fi
%    \end{macrocode}
% We use |\otf@makeglobal| from \pkg{otfontdef} to ``export'' the definitions
% that are needed globally.
%    \begin{macrocode}
\otf@makeglobal{Mn@option@opticals}
\otf@makeglobal{Mn@option@fontset}
\ifx\@nodocument\relax\else
  \PackageInfo{MinionPro-FontDef}{%
    Configuration:\space\Mn@option@fontset,\space\Mn@option@opticals,\space
    normalsize=\the\Mn@option@normalsize}%
\fi
%    \end{macrocode}
%
% \subsubsection{Configuration database}
%
%    \begin{macrocode}
\newcount\Mn@config@cnt
\Mn@config@cnt=0
\newcommand\Mn@curr@config{Mn@config@\romannumeral\Mn@config@cnt}
%    \end{macrocode}
% These commands help in setting up the configuration database. They do not
% need to be global. But the config database itself has to be.
%
% |#3| is added to all instances listed in |#2| of configuration class |#1|.
% |#3| is read with \acro{NFSS} catcodes.
%    \begin{macrocode}
\newcommand\Mn@AddToConfig{%
  \begingroup
  \nfss@catcodes
  \expandafter\endgroup
  \Mn@AddToConfig@
}
\newcommand\Mn@AddToConfig@[3]{%
  \advance\Mn@config@cnt\@ne
  \@namedef{\Mn@curr@config}{#3}%
  \otf@makeglobal{\Mn@curr@config}
%<debug&show>\expandafter\show\csname\Mn@curr@config\endcsname
  \@for\Mn@tempa:=#2\do{%
    \@ifundefined{Mn@config@#1@\Mn@tempa}{%
      \@temptokena{}%
    }{%
      \@temptokena\expandafter\expandafter\expandafter
        {\csname Mn@config@#1@\Mn@tempa\endcsname}%
    }%
    \@expandtwoargs\@namedef{Mn@config@#1@\Mn@tempa}{%
      \the\@temptokena
      \expandafter\noexpand\csname\Mn@curr@config\endcsname
    }%
    \otf@makeglobal{Mn@config@#1@\Mn@tempa}% perhaps defer to only execute once
%<debug&show>\expandafter\show\csname Mn@config@#1@\Mn@tempa\endcsname
  }%
}
%    \end{macrocode}
% Let us look at an example of how the configuration database looks internally for
% $(|shape|, |sw|)$, which is specified below in three steps. The following
% lines show different depths of expansion of the macro |\Mn@config@shape@sw|,
% which finally yields the complete configuration:
% \exampleconfig{shape}{sw}
%
% The following commands are used in the |Declare...Family| commands to access
% the previously built configuration database. They must be expandable. |#3| is
% used as a default if no entry is found in the database.
%    \begin{macrocode}
\newcommand*\Mn@UseConfig[2]{%
  \Mn@UseConfigOrDefault{#1}{#2}{}%
}
\newcommand*\Mn@UseConfigOrDefault[3]{%
  \@ifundefined{Mn@config@#1@#2}{#3}%
    {\@nameuse{Mn@config@#1@#2}}%
}
\newcommand*\Mn@TheConfig[2]{%
  \@ifundefined{Mn@config@#1@#2}{}{%
    \expandafter\noexpand\csname Mn@config@#1@#2\endcsname
  }%
}
\otf@makeglobal{Mn@UseConfig}
\otf@makeglobal{Mn@UseConfigOrDefault}
\otf@makeglobal{Mn@TheConfig}
%    \end{macrocode}
% Here comes the configuration.
%    \begin{macrocode}
\Mn@AddToConfig{opticals}{opticals}{
      <-8.5>  otf* [optical=Capt]
   <8.5-13.1> otf* [optical=Text]
  <13.1-20>   otf* [optical=Subh]
    <20->     otf* [optical=Disp]
}
\Mn@AddToConfig{opticals}{noopticals}{
      <->     otf* [optical=Text]
}
\Mn@AddToConfig{opticals}{slides}{
      <->     otf* [optical=Capt]
}
%    \end{macrocode}
%
%    \begin{macrocode}
\ifdim\Mn@option@normalsize<10.1pt
  \Mn@AddToConfig{fontset/weight}{fullfamily/m}{
        <-6>    otf* [weight=Semibold]
       <6-8.5>  otf* [weight=Medium]
     <8.5->     otf* [weight=Regular]
  }
\else
  \Mn@AddToConfig{fontset/weight}{fullfamily/m}{
        <-6>    otf* [weight=Semibold]
       <6-10.1> otf* [weight=Medium]
    <10.1->     otf* [weight=Regular]
  }
\fi
\Mn@AddToConfig{fontset/weight}{medfamily/m}{
      <-6>    otf* [weight=Semibold]
     <6->     otf* [weight=Regular]
}
\Mn@AddToConfig{fontset/weight}{smallfamily/m}{
      <->     otf* [weight=Regular]
}
%
\Mn@AddToConfig{fontset/weight}{fullfamily/b,medfamily/b}{
      <-6>    otf* [weight=Bold]
     <6->     otf* [weight=Semibold]
}
\Mn@AddToConfig{fontset/weight}{smallfamily/b}{
      <->     otf* [weight=Bold]
}
%
\Mn@AddToConfig{weight}{eb}{
      <->     otf* [weight=Bold]
}
%    \end{macrocode}
%
%    \begin{macrocode}
\Mn@AddToConfig{shape}{ssc,sscit}{
      <->     otf* [spacing=l2]
}
\Mn@AddToConfig{shape}{n,it,sw,sc,scit}{
      <-8>    otf* [spacing=l1]
}
\Mn@AddToConfig{encoding/shape}{U/n,U/it}{
      <->     otf* [spacing=]
}
%
\Mn@AddToConfig{shape}{sc,ssc,scit,sscit}{
      <->     otf* [variant=sc]
}
\Mn@AddToConfig{shape}{sw}{
      <->     otf* [variant=swash]
}
%    \end{macrocode}
%
%    \begin{macrocode}
\Mn@AddToConfig{shape}{it,scit,sscit,sw}{
      <->     otf* MinionPro-It
}
\Mn@AddToConfig{shape}{n,sc,ssc}{
      <->     otf* MinionPro
}
\Mn@AddToConfig{encoding/shape}{OML/it}{
      <->     otf* [figures=] MinionPro-Mixed
}
\Mn@AddToConfig{encoding/shape}{OML/n}{
      <->     otf* [figures=] MinionPro-French
}
%    \end{macrocode}
% Substitutions
%    \begin{macrocode}
\Mn@AddToConfig{sub:series} {sb}    {b}
\Mn@AddToConfig{sub:series} {bx}    {b}
\Mn@AddToConfig{sub:shape}  {sl}    {it}
\Mn@AddToConfig{sub:shape}  {scsl}  {scit}
\Mn@AddToConfig{sub:shape}  {sscsl} {sscit}
\Mn@AddToConfig{sub:shape}  {scsw}  {scit}
\Mn@AddToConfig{sub:shape}  {sscsw} {sscit}
\Mn@AddToConfig{sub:encoding/shape}{TS1/sw}{it}
%    \end{macrocode}
% Code for the last argument of |\DeclareFontShape|
%    \begin{macrocode}
\Mn@AddToConfig{code:shape}{sw}{
  \skewchar\font='337
}
%\Mn@AddToConfig{code:shape}{ssc,sscit}{
%  \Mn@adjust@fontdimen2{\Mn@interword@fct} % interword space
%  \Mn@adjust@fontdimen3{\Mn@interword@fct} % interword stretch
%  \Mn@adjust@fontdimen4{\Mn@interword@fct} % interword shrink
%  \Mn@adjust@fontdimen7{\Mn@interword@fct} % extra space
%}
%    \end{macrocode}
% This function is currently used to scale all four components of interword
% space: space, stretch, shrink, and extra space.
%    \begin{macrocode}
%\newcommand*\Mn@interword@fct{%
%  \Mn@fontdimen=1.15\Mn@fontdimen
%}
%\otf@makeglobal{Mn@interword@fct}
%    \end{macrocode}
%
% \subsubsection{Declaration of font families and shapes}
%
%    \begin{macrocode}
\newcommand*\Mn@DeclareFontShape[6][]{%
%    \end{macrocode}
% Check if any substitutions are specified.
%    \begin{macrocode}
  \edef\@tempa{%
    \Mn@UseConfig{sub:series}{#4}%
    \Mn@UseConfigOrDefault{sub:encoding/shape}{#2/#5}{%
      \Mn@UseConfig{sub:shape}{#5}}%
  }%
  \ifx\@tempa\@empty
%    \end{macrocode}
% Collect the configuration and declare the font shape. |\DeclareFontShape|
% fully expands its fifth argument (with our macros |\Mn@UseConfig| in it), but
% we have to retrieve the code for the sixth argument ourselves.
%    \begin{macrocode}
    \@temptokena={%
      \DeclareFontShape{#2}{#3-#6}{#4}{#5}{%
        \Mn@UseConfig{opticals}      {\Mn@option@opticals}%
        \Mn@UseConfig{fontset/weight}{\Mn@option@fontset/#4}%
        \Mn@UseConfig{weight}        {#4}%
        \Mn@UseConfig{encoding/shape}{#2/#5}%
        \Mn@UseConfig{shape}         {#5}%
      }}%
    \edef\@tempa{\the\@temptokena{\Mn@TheConfig{code:shape}{#5}}}%
    \@tempa
  \else
%    \end{macrocode}
% Generate the substitution. (All substitutions are silent at the moment.)
%    \begin{macrocode}
    \DeclareFontShape{#2}{#3-#6}{#4}{#5}{%
      <->ssub*#3-#6%
      /\Mn@UseConfigOrDefault{sub:series}{#4}{#4}%
      /\Mn@UseConfigOrDefault{sub:encoding/shape}{#2/#5}{%
        \Mn@UseConfigOrDefault{sub:shape}{#5}{#5}}%
    }{}%
  \fi
}
\otf@makeglobal{Mn@DeclareFontShape}
\otf@makeglobal{\string\Mn@DeclareFontShape}
%    \end{macrocode}
% |#2| contains the encoding, |#3| the family, and |#1| a list of figure
% versions (or |Extra|).
%    \begin{macrocode}
\newcommand*\Mn@DeclareLargeFontFamily[3][LF,OsF,TLF,TOsF]{%
  \Mn@DeclareFontFamily{#1}{#2}{#3}
    {m,sb,b,bx,eb} {n,it,sc,ssc,scit,sscit,sw,scsl,scsw,sscsl,sscsw,sl}%
}
\newcommand*\Mn@DeclareSmallFontFamily[3][LF,OsF,TLF,TOsF]{%
  \Mn@DeclareFontFamily{#1}{#2}{#3}
    {m,sb,b,bx,eb} {n,it,sl}%
}
\newcommand*\Mn@DeclareMathFontFamily[3][TOsF]{%
  \Mn@DeclareFontFamily[\skewchar\font=255]{#1}{#2}{#3}
    {m,sb,b,bx,eb} {n,it}%
}
%    \end{macrocode}
% An additional macro |\csname\string\foo\endcsname| is generated by
% |\newcommand| for processing an optional argument of |\foo|.
%    \begin{macrocode}
\otf@makeglobal{Mn@DeclareLargeFontFamily}
\otf@makeglobal{\string\Mn@DeclareLargeFontFamily}
\otf@makeglobal{Mn@DeclareSmallFontFamily}
\otf@makeglobal{\string\Mn@DeclareSmallFontFamily}
\otf@makeglobal{Mn@DeclareMathFontFamily}
\otf@makeglobal{\string\Mn@DeclareMathFontFamily}
%    \end{macrocode}
%
%    \begin{macrocode}
\newcommand*\Mn@DeclareFontFamily[6][]{%
  \@for\Mn@variant:=#2\do{%
    \DeclareFontFamily {#3}{#4-\Mn@variant}{#1}%
  }%
  \Mn@DeclareFontShapes{#3}{#4}
    {#5} {#6} {#2}%
}
\otf@makeglobal{Mn@DeclareFontFamily}
\otf@makeglobal{\string\Mn@DeclareFontFamily}
%    \end{macrocode}
%
%    \begin{macrocode}
\newcommand*\Mn@DeclareFontShapes[5]{%
  \@for\Mn@series:=#3\do{%
    \@for\Mn@shape:=#4\do{%
      \@for\Mn@variant:=#5\do{%
        \Mn@DeclareFontShape{#1}{#2}{\Mn@series}{\Mn@shape}{\Mn@variant}%
      }%
    }%
  }%
}
\otf@makeglobal{Mn@DeclareFontShapes}
%    \end{macrocode}
% Adjust font dimension |#1| of the current font. The function in |#2| should
% replace the old value in dimen |\Mn@fontdimen| with a new one (which may
% depend on other parameters like |\f@size|).
%    \begin{macrocode}
\newdimen\Mn@fontdimen
\newcommand*\Mn@adjust@fontdimen[2]{%
  \Mn@fontdimen=\fontdimen#1\font
  #2%
  \fontdimen#1\font=\Mn@fontdimen
}
\otf@makeglobal{Mn@adjust@fontdimen}
%    \end{macrocode}
%
%    \begin{macrocode}
\ifx\@nodocument\relax
  \endgroup
\fi
%    \end{macrocode}
%    \begin{macrocode}
%<*debug>
\newcommand\old@DeclareFontFamily{}
\let\old@DeclareFontFamily\DeclareFontFamily
\renewcommand\DeclareFontFamily[3]{
  \begingroup\escapechar`\\%
  \edef\@tempa{\noexpand\DeclareFontFamily{#1}{#2}}%
  \@temptokena\expandafter{\@tempa{#3}}%
  \message{\the\@temptokena}%
  \endgroup
  \old@DeclareFontFamily{#1}{#2}{#3}%
}
\newcommand\old@DeclareFontShape{}
\let\old@DeclareFontShape\DeclareFontShape
\renewcommand\DeclareFontShape[6]{
  \begingroup\escapechar`\\%
  \edef\@tempa{\noexpand\DeclareFontShape{#1}{#2}{#3}{#4}{#5}}%
  \@temptokena\expandafter{\@tempa{#6}}%
  \message{\the\@temptokena}%
  \endgroup
  \old@DeclareFontShape{#1}{#2}{#3}{#4}{#5}{#6}%
}
%</debug>
%    \end{macrocode}
%
% We define font family aliases so that we can place all configurations for
% the MinionPro family variants into one \pkg{microtype} file:
% \file{mt-MinionPro.cfg}. We use \pkg{microtype}'s hook if \pkg{microtype}
% has not been loaded yet (which should be the case); otherwise we can
% execute the alias definitions directly.
%    \begin{macrocode}
\gdef\Mn@MicroType@Aliases{%
  \DeclareMicrotypeAlias{MinionPro-LF}{MinionPro}%
  \DeclareMicrotypeAlias{MinionPro-OsF}{MinionPro}%
  \DeclareMicrotypeAlias{MinionPro-TLF}{MinionPro}%
  \DeclareMicrotypeAlias{MinionPro-TOsF}{MinionPro}%
}
\@ifundefined{Microtype@Hook}{%
  \global\let\Microtype@Hook\Mn@MicroType@Aliases
}{%
  \g@addto@macro\Microtype@Hook{\Mn@Microtype@Aliases}%
}%
\@ifundefined{DeclareMicroTypeAlias}{}{\Mn@MicroType@Aliases}%
%</fontdef>
%    \end{macrocode}
%
% Using these macros the various \acro{FD} files become simple one-liners.
%
%    \begin{macrocode}
%<*fd>
%<*gobble>
\errmessage{No printable documentation yet}
%</gobble>
%<*fontdef>
% trying to make this package loadable in an \acro{FD} file (not in the
% preamble)
\ProvidesFile{MinionPro-FontDef.sty}
  [2005/04/13 v2.1 font definition macros]
%
% allow this file to be included multiple times
\ifx\Mn@DeclareFontShape\@undefined\else\endinput\fi

\ifx\@nodocument\relax
  % \iffalse
%<*driver>
\ProvidesFile{otfontdef.dtx}
%</driver>
%<otfd>\@ifundefined{@nodocument}
%<otfd>  {\ProvidesFile{otfontdef.sty}}
%<otfd>  {\NeedsTeXFormat{LaTeX2e}\ProvidesPackage{otfontdef}}
%<*otfd|driver>
  [2005/05/05 v0.1c OpenType font definition]
%</otfd|driver>
%<*driver>
\documentclass{ltxdoc}
\newcommand\pkg[1]{\textsf{#1}}
\newcommand\acro[1]{#1}
\RecordChanges
\begin{document}
\DocInput{otfontdef.dtx}
\end{document}
%</driver>
%<*otfd>
% \fi
% 
% \GetFileInfo{otfontdef.dtx}
% 
% \title{OpenType Font Definition}
% \author{Andreas B\"uhmann}
% \date{\fileversion\ -- \filedate}
% \maketitle
% 
% \begin{abstract}
%   The package \pkg{otfontdef} provides a simple but powerful key-value-based interface for
%   font definition files. It enables you to vary multiple font
%   attributes according to the font size separately from each other.
% \end{abstract}
%
% \section{Motivation}
%
% Writing font definition files for professional OpenType fonts is tedious.
% You have many different aspects like optical size, weight, and letterspacing
% at your disposal that you might want to vary according to the font size but
% each one independently of the other. To achieve this with ordinary means
% involves intersecting size intervals, copying unchanged parts, etc., which
% leads to a lot of repetition and, in turn, to decreased maintainability.
% 
% As an example, imagine that you are about to map the font
% |T1/MinionPro-OsF/m/it| to external fonts in a font definition file. This is
% easy if there is exactly one external font:
% \begin{verbatim}
% \DeclareFontShape{T1}{MinionPro-OsF}{m}{it}{
%       <->     MinionPro-It-osf-t1
% }{}
% \end{verbatim}
% But let us assume that you have access to multiple optical sizes of that
% font and that you would like to use them (for simplicity, let us use only
% two optical sizes, Caption and Text):
% \begin{verbatim}
%       <-8.5>  MinionPro-ItCapt-osf-t1
%    <8.5->     MinionPro-It-osf-t1
% \end{verbatim}
% You then decide that it would be nice to increase the weights at small
% sizes; for instance, use Medium for sizes less than 8\,pt and Semibold for
% sizes less than 6\,pt. This implies having to split the first interval
% twice:
% \begin{verbatim}
%       <-6>    MinionPro-SemiboldItCapt-osf-t1
%      <6-8>    MinionPro-MediumItCapt-osf-t1
%      <8-8.5>  MinionPro-ItCapt-osf-t1
% \end{verbatim}
% If you finally come up with the idea of using a letterspaced version of your
% font at small sizes (less than 7\,pt), you will arrive at this
% configuration, where each of your design decisions is scattered over up to
% five lines:
% \begin{verbatim}
% \DeclareFontShape{T1}{MinionPro-OsF}{m}{it}{
%       <-6>    MinionPro-SemiboldItCapt-osf-l1-t1
%      <6-7>    MinionPro-MediumItCapt-osf-l1-t1
%      <6-8>    MinionPro-MediumItCapt-osf-t1
%      <8-8.5>  MinionPro-ItCapt-osf-t1
%    <8.5->     MinionPro-It-osf-t1
% }{}
% \end{verbatim}
% This package enables you to specify these decisions directly.
% \begin{verbatim}
% \DeclareFontShape{T1}{MinionPro-OsF}{m}{it}{
%       <->     otf* MinionPro-It
%       <-8.5>  otf* [optical=Capt]
%       <-6>    otf* [weight=Semibold]
%      <6-8>    otf* [weight=Medium]
%       <-7>    otf* [spacing=l1]
% }{}
% \end{verbatim}
% The package will then analyze your specifications, combine them, and
% generate the correct external font names according to some naming scheme.
%
% \section{Usage}
% 
% \dots
%
% \StopEventually{}
%
% \section{Implementation}
%
% \begin{macro}{\otf@disable@preamblecmds}
%   In a second we need to temporarily disable all commands that can be used
%   in the preamble only (and that occur in the current version of keyval).
%    \begin{macrocode}
\newcommand\otf@disable@preamblecmds{%
  \def\@gobble@optional{%
    \@ifnextchar[\@gobble@optional@{}%]
  }%
  \def\@gobble@optional@[##1]{}%
  \def\NeedsTeXFormat##1{\@gobble@optional}%
  \def\ProvidesPackage##1{\@gobble@optional}%
  \let\DeclareOption\@gobbletwo
  \let\ExecuteOptions\@gobble
  \def\ProcessOptions{\@ifstar\@gobble{}}%
}
%    \end{macrocode}
% \end{macro}
% Try hard to also work when loaded from inside an \acro{FD} file. We use
% |\space| because literal spaces are ignored.
%    \begin{macrocode}
\ifx\@nodocument\relax
  \PackageWarningNoLine{otfontdef}{Please\space load\space me\space
  in\space the\space preamble.\MessageBreak
  I'm\space doing\space my\space best\space to\space continue\space anyway}%
  \@ifundefined{define@key}{
    \begingroup
%    \end{macrocode}
% \pkg{keyval} uses the space token to define its commands. We make sure that
% it is the right one. (Space has catcode 9 (ignore) in \acro{FD} files.)
%    \begin{macrocode}
    \catcode32=10
    \otf@disable@preamblecmds
%    \end{macrocode}
% I have always waited to find a use for |\globaldefs|; here it is. It is
% needed for the definitions in \pkg{keyval} to survive this group and
% especially the group around the \acro{FD} file. We must be extremely careful
% not to execute definitions that we do not want to be global: Hence, we cannot
% let |\ProvidesPackage| to |\ProvidesFile| because it changes catcodes. Let us
% hope that \pkg{keyval} does not change.
%    \begin{macrocode}
    \globaldefs=1
    \input keyval.sty
    \endgroup
  }{}%
\else
  \RequirePackage{keyval}[1999/03/16 v1.13]
\fi
%    \end{macrocode}
%
% \begin{macro}{\otf@makeglobal}
%   We have to make definitions global to allow this package to be used from
%   inside \acro{FD} files,
%    \begin{macrocode}
\newcommand\otf@makeglobal[1]{
  \global\expandafter\let\csname #1\expandafter\endcsname
  \csname #1\endcsname
}
%    \end{macrocode}
% but we need not when loaded as a normal package.
%    \begin{macrocode}
\ifx\@nodocument\relax\else
  \let\otf@makeglobal\@gobble
\fi
%    \end{macrocode}
% \end{macro}
% \begin{macro}{\otf@info}
% We only show information on chosen configurations when asked to do so.
% \changes{v0.2}{2005/04/19}{Introduce trace option}
%    \begin{macrocode}
\newif\ifotf@trace\otf@tracefalse
\newcommand*\otf@info[1]{}
\ifx\@nodocument\relax\else
  \DeclareOption{trace}{\otf@tracetrue}
  \ProcessOptions\relax
  \ifotf@trace
    \def\otf@info#1{\PackageInfo{otfontdef}{#1}}
  \fi
\fi
\otf@makeglobal{otf@info}
%    \end{macrocode}
% \end{macro}
% \begin{macro}{\ifotf@options}
%   Whether we are looking for options or not when scanning the configuration.
%    \begin{macrocode}
\newif\ifotf@options
\otf@optionsfalse
\otf@makeglobal{ifotf@options}
\otf@makeglobal{otf@optionstrue}
\otf@makeglobal{otf@optionsfalse}
%    \end{macrocode}
% \end{macro}
% 
% \begin{macro}{\otf@keys}
% \begin{macro}{\otf@definekey}
%   Only the first occurrence of a key sets the associated value; other
%   occurrences are ignored.
%    \begin{macrocode}
\newcommand\otf@keys{}
\newcommand\otf@definekey[1]{%
  \define@key{otf}{#1}{%
    \@ifundefined{otf@@#1}{%
      \@namedef{otf@@#1}{##1}%
    }{}%
  }%
  \expandafter\let\csname otf@@#1\endcsname\relax
  \g@addto@macro\otf@keys{\do{#1}}%
  \otf@makeglobal{otf@@#1}%
  \otf@makeglobal{KV@otf@#1}%
}
%    \end{macrocode}
% \end{macro}
% \end{macro}
% 
% \begin{macro}{\otf@default@keys}
% \begin{macro}{\otf@definedefault}
%    \begin{macrocode}
\newcommand\otf@default@keys{}
\newcommand\otf@definedefault[2]{%
  \@namedef{KV@otf@#1@default}{#2}%
  \edef\otf@default@keys{\otf@default@keys,#1}%
  \otf@makeglobal{KV@otf@#1@default}
}
%    \end{macrocode}
% \end{macro}
% \end{macro}
% Define all font attributes
%    \begin{macrocode}
\otf@definekey{family}
\otf@definekey{weight}
\otf@definekey{shape}
\otf@definekey{optical}
\otf@definekey{variant}
\otf@definekey{figures}
\otf@definekey{spacing}
\otf@definekey{encoding}
\otf@definekey{size}
\otf@makeglobal{otf@keys}
%    \end{macrocode}
% and how their defaults are determined (if at all).
%    \begin{macrocode}
\otf@definedefault{weight}
  {\expandafter\KV@otf@weight\expandafter{\otf@Regular}}
\otf@definedefault{optical}
  {\expandafter\KV@otf@optical\expandafter{\otf@Text}}
\otf@definedefault{variant}
  {\expandafter\otf@splitname@int\f@family--\@empty}
\otf@definedefault{figures}
  {\expandafter\otf@splitname@int\f@family--\@empty}
\otf@definedefault{encoding}
  {\expandafter\KV@otf@encoding\expandafter{\f@encoding}}
\otf@definedefault{size}
  {\expandafter\KV@otf@size\expandafter{\f@size}}
\otf@definedefault{shape}
  {\expandafter\KV@otf@shape\expandafter{\otf@Regular}}
\otf@makeglobal{otf@default@keys}
%    \end{macrocode}
% 
% \begin{macro}{\otf@Regular}
% \begin{macro}{\otf@Text} 
%   All characters of these strings must have catcode 12 (other). They will be
%   matched against substrings of |\DeclareFontShape| configurations.
%    \begin{macrocode}
\newcommand*\otf@Regular{Regular}
\newcommand*\otf@Text{Text}
\newcommand*\otf@Ornaments{Ornaments}
\@onelevel@sanitize\otf@Regular
\@onelevel@sanitize\otf@Text
\@onelevel@sanitize\otf@Ornaments
\otf@makeglobal{otf@Regular}
\otf@makeglobal{otf@Text}
\otf@makeglobal{otf@Ornaments}
%    \end{macrocode}
% \end{macro}
% \end{macro}
% We register a new size function |otf|, which can then be used in \acro{FD}
% files. We procede in two phases when building the configuration for a
% requested font: In the first phase, we ignore all |otf| entries that give
% only options (no |\mandatory@arg|); from the first entry that has a
% mandatory argument we start into the second phase, where we rescan the
% configuration looking for options.
%    \begin{macrocode}
\ifx\@nodocument\relax
  \begingroup
  \def\DeclareSizeFunction#1#2{\endgroup\global\@namedef{s@fct@#1}{#2}}%
  \expandafter
\fi
\DeclareSizeFunction{otf}{%
  \ifotf@options
    \otf@get@options
  \else
    \ifx\mandatory@arg\@empty\else
      \otf@get@external@font
    \fi
  \fi
}
%    \end{macrocode}
%
% \begin{macro}{\otf@get@options}
%   Simply process all key-value pairs given in the optional argument.
%    \begin{macrocode}
\newcommand\otf@get@options{%
  \@expandtwoargs\setkeys{otf}{\optional@arg}%
}
\otf@makeglobal{otf@get@options}
%    \end{macrocode}
% \end{macro}
%
% \begin{macro}{\otf@splitname@ext}
% \begin{macro}{\otf@splitname@int}
% \changes{v0.2}{2005/05/05}{Remove erroneous space token}
%    \begin{macrocode}
\newcommand\otf@splitname@ext{}
\def\otf@splitname@ext#1-#2-#3\@empty{%
  \def\otf@@family{#1}%
  \def\otf@@shape{#2}%
  \ifx\otf@@shape\@empty
    \let\otf@@shape\relax
  \fi
}
\newcommand\otf@splitname@int{}
\def\otf@splitname@int#1-#2-#3\@empty{%
  \KV@otf@family{#1}%
  \def\@tempa{#2}%
  \ifx\@tempa\otf@Ornaments
    \KV@otf@variant{orn}%
  \else
    \KV@otf@figures{#2}%
  \fi
}
\otf@makeglobal{otf@splitname@ext}
\otf@makeglobal{otf@splitname@int}
%    \end{macrocode}
% \end{macro}
% \end{macro}
%
% \begin{macro}{\otf@get@external@font}
%   This is the master macro that coordinates the processing. We first
%   determine family and (possibly) shape from the mandatory argument.
%    \begin{macrocode}
\newcommand\otf@get@external@font{%
  \expandafter\otf@splitname@ext\mandatory@arg--\@empty
%    \end{macrocode}
% We can then rescan the current configuration for further options.
%    \begin{macrocode}
  \otf@optionstrue
  \try@size@range
%    \end{macrocode}
% If any attribute has not received a value, we use its default.
%    \begin{macrocode}
  \@expandtwoargs\setkeys{otf}{\otf@default@keys}%
%    \end{macrocode}
% After having informed the user about the values we have collected, we build
% the external font name by applying the appropriate naming scheme.
%    \begin{macrocode}
  \begingroup
  \def\do##1{\otf@showoption{##1}\MessageBreak}%
  \otf@info{Using\space configuration\MessageBreak
    \otf@keys for\space font\space\font@name}%
  \endgroup
  \@ifundefined{otf@scheme@@\otf@@family}{\otf@scheme@default}{%
    \@nameuse{otf@scheme@@\otf@@family}%
  }%
  \otf@info{Trying\space to\space load\space external\space font\MessageBreak
    `\external@font'}%
}
\otf@makeglobal{otf@get@external@font}
%    \end{macrocode}
% \end{macro}
% 
% \begin{macro}{\otf@scheme@default}
%   This is the font naming scheme used in the MinionPro project.
%   \changes{v0.2}{2005/04/20}{Improve grouping}
%    \begin{macrocode}
\newcommand\otf@head{}
\newcommand\otf@tail{}
\newcommand\otf@scheme@default{%
  \begingroup
  \edef\@tempa{\lowercase{\def\noexpand\otf@tail{%
    \otf@opt\otf@@figures
    \otf@opt\otf@@variant
    \otf@opt\otf@@spacing
    \otf@@encoding
  }}}\@tempa
  \edef\otf@head{%
    \ifx\otf@@weight\otf@Regular\else\otf@@weight\fi
    \otf@format@shape\otf@@shape
    \ifx\otf@@optical\otf@Text\else\otf@@optical\fi}%
  \ifx\otf@head\@empty
    \edef\otf@head{\otf@@family-\otf@Regular}%
  \else
    \edef\otf@head{\otf@@family-\otf@head}%
  \fi
  \edef\@tempa{\endgroup\def\noexpand\external@font{%
    \otf@head-\otf@tail\space at\space\otf@@size pt}}%
  \@tempa
}
\otf@makeglobal{otf@scheme@default}
\newcommand\otf@format@shape[1]{%
  \@ifundefined{otf@format@shape@#1}{}{\@nameuse{otf@format@shape@#1}}%
}
\newcommand\otf@format@shape@Regular{}%
\newcommand\otf@format@shape@Italic{It}%
\newcommand\otf@format@shape@It{It}%
\otf@makeglobal{otf@format@shape}
\otf@makeglobal{otf@format@shape@Regular}
\otf@makeglobal{otf@format@shape@Italic}
\otf@makeglobal{otf@format@shape@It}
%    \end{macrocode}
% \end{macro}
% \begin{macro}{\DeclareFontNamingScheme}
%   We do not make this declaration command global. The who wants to use it
%   should really load the package in preamble.
%   \changes{v0.2}{2005/04/20}{Use the term `scheme' instead of `pattern'}
%    \begin{macrocode}
\newcommand*\DeclareFontNamingScheme[1]{%
  \@namedef{otf@scheme@@#1}%
}
%    \end{macrocode}
% \end{macro}
%
% \begin{macro}{\otf@opt}
%   An optional name component, separated from the following ones by a hyphen.
%    \begin{macrocode}
\newcommand*\otf@opt[1]{%
  \ifx\relax#1\@empty\else\if\@empty#1\@empty\else#1-\fi\fi
}
%    \end{macrocode}
% \end{macro}
% \begin{macro}{\otf@showoption}
%   Format an option and its current value for displaying it.
%    \begin{macrocode}
\newcommand*\otf@showoption[1]{%
  \@spaces #1\space=\space\@ifundefined{otf@@#1}{<undefined>}{%
    \expandafter\expandafter\expandafter\strip@prefix
    \expandafter\meaning\csname otf@@#1\endcsname}
}
\otf@makeglobal{otf@opt}
\otf@makeglobal{otf@showoption}
%    \end{macrocode}
% \end{macro}
%
% \PrintChanges
% \Finale
%
% \iffalse
%</otfd>
% \fi
\endinput

\else
  \NeedsTeXFormat{LaTeX2e}
  \RequirePackage{otfontdef}
\fi

% Reset |\escapechar| (-1 in \acro{FD} files) to make |\newcommand| work. The
% additional group does not harm then; we have to make the important things
% global anyway. 
\ifx\@nodocument\relax
  \begingroup\escapechar`\\
\fi

\newcommand\Mn@option@opticals{opticals}
\newcommand\Mn@option@family{medfamily}
\newdimen\Mn@option@normalsize
\global\Mn@option@normalsize10pt
\otf@makeglobal{Mn@option@opticals}
\otf@makeglobal{Mn@option@family}

\ifx\@nodocument\relax\else
  \DeclareOption{slides}     {\let\Mn@option@opticals\CurrentOption}
  \DeclareOption{opticals}   {\let\Mn@option@opticals\CurrentOption}
  \DeclareOption{noopticals} {\let\Mn@option@opticals\CurrentOption}
  \DeclareOption{smallfamily}{\let\Mn@option@family\CurrentOption}
  \DeclareOption{medfamily}  {\let\Mn@option@family\CurrentOption}
  \DeclareOption{fullfamily} {\let\Mn@option@family\CurrentOption}

  \DeclareOption{normalsize} {%
    % idea stolen from microtype.dtx
    \begingroup
    \def\set@fontsize#1#2#3#4\@nil{%
      \@defaultunits\global\Mn@option@normalsize#2pt\relax\@nnil}%
    \normalsize\@nil
    \endgroup
  }

  \ExecuteOptions{fullfamily,opticals}
  \ProcessOptions\relax
\fi

\newcount\Mn@config@cnt
\Mn@config@cnt=0
\newcommand\Mn@curr@config{Mn@config@\romannumeral\Mn@config@cnt}

% These commands help in setting up the configuration database; they do not
% need to be global. But the config database itself does.
\newcommand\Mn@AddToConfig{%
  \begingroup
  \nfss@catcodes
  \expandafter\endgroup
  \Mn@AddToConfig@
}
\newcommand\Mn@AddToConfig@[3]{%
  \advance\Mn@config@cnt\@ne
  \@namedef{\Mn@curr@config}{#3}%
  \otf@makeglobal{\Mn@curr@config}
%  \expandafter\show\csname\Mn@curr@config\endcsname
  \@for\Mn@tempa:=#2\do{%
    \@ifundefined{Mn@config@#1@\Mn@tempa}{%
      \@temptokena{}%
    }{%
      \@temptokena\expandafter\expandafter\expandafter
        {\csname Mn@config@#1@\Mn@tempa\endcsname}%
    }%
    \@expandtwoargs\@namedef{Mn@config@#1@\Mn@tempa}{%
      \the\@temptokena
      \expandafter\noexpand\csname\Mn@curr@config\endcsname
    }%
    \otf@makeglobal{Mn@config@#1@\Mn@tempa}% perhaps defer to only execute once
%    \expandafter\show\csname Mn@config@#1@\Mn@tempa\endcsname
  }%
}

% This command is used in the Declare...Family commands
\newcommand\Mn@UseConfig[2]{%
  \@ifundefined{Mn@config@#1@#2}{}%
    {\@nameuse{Mn@config@#1@#2}}%
}
\otf@makeglobal{Mn@UseConfig}

% Here comes the configuration.
\Mn@AddToConfig{opticals}{opticals}{
      <-8.5>  otf* [optical=Capt]
   <8.5-13.1> otf* [optical=Text]
  <13.1-20>   otf* [optical=Subh]
    <20->     otf* [optical=Disp]
}
\Mn@AddToConfig{opticals}{noopticals}{
      <->     otf* [optical=Text]
}
\Mn@AddToConfig{opticals}{slides}{
      <->     otf* [optical=Capt]
}

\ifdim\Mn@option@normalsize<10.1pt
  \Mn@AddToConfig{family/weight}{fullfamily/m}{
        <-6>    otf* [weight=Semibold]
       <6-8.5>  otf* [weight=Medium]
     <8.5->     otf* [weight=Regular]
  }
\else
  \Mn@AddToConfig{family/weight}{fullfamily/m}{
        <-6>    otf* [weight=Semibold]
       <6-10.1> otf* [weight=Medium]
    <10.1->     otf* [weight=Regular]
  }
\fi
\Mn@AddToConfig{family/weight}{medfamily/m}{
      <-6>    otf* [weight=Semibold]
     <6->     otf* [weight=Regular]
}
\Mn@AddToConfig{family/weight}{smallfamily/m}{
      <->     otf* [weight=Regular]
}

\Mn@AddToConfig{family/weight}{fullfamily/sb,medfamily/sb}{
      <-6>    otf* [weight=Bold]
     <6->     otf* [weight=Semibold]
}
\Mn@AddToConfig{family/weight}{smallfamily/sb}{
      <->     otf* [weight=Bold]
}

\Mn@AddToConfig{weight}{b}{
      <->     otf* [weight=Bold]
}

\Mn@AddToConfig{shape}{ssc,sscit}{
      <->     otf* [spacing=l2]
}
\Mn@AddToConfig{shape}{n,it,sw,sc,scit}{
      <-8>    otf* [spacing=l1]
}
\Mn@AddToConfig{shape/math}{n,it,sc,scit}{
      <->     otf* [spacing=l1]
}

\Mn@AddToConfig{shape}{sc,ssc,scit,sscit}{
      <->     otf* [variant=sc]
}
\Mn@AddToConfig{shape}{sw}{
      <->     otf* [variant=swash]
}

\Mn@AddToConfig{shape}{it,scit,sscit,sw}{
      <->     otf* MinionPro-It
}
\Mn@AddToConfig{shape}{n,sc,ssc}{
      <->     otf* MinionPro
}

\newcommand\Mn@DeclareFontShape[6][]{%
  \DeclareFontShape{#2}{#3}{#4}{#5}{%
    \Mn@UseConfig{opticals}     {\Mn@option@opticals}%
    \Mn@UseConfig{family/weight}{\Mn@option@family/#4}%
    \Mn@UseConfig{weight}       {#4}%
    \Mn@UseConfig{shape}        {#5}%
    \Mn@UseConfig{shape/#1}     {#5}%
  }{#6}%
}
\otf@makeglobal{Mn@DeclareFontShape}
\otf@makeglobal{\string\Mn@DeclareFontShape}

% #1 encoding #2 MinionPro-figures (#3/#4 only for backwards compatibility)
\newcommand*\Mn@DeclareLargeFontFamily[4]{%
  \Mn@DeclareSmallFontFamily  {#1}{#2}{}{}%
  \DeclareFontFamily          {#1}{#2}{}%
  \Mn@DeclareFontShape        {#1}{#2}{m}{sc}    {}%
  \Mn@DeclareFontShape        {#1}{#2}{m}{ssc}   {}%
  \Mn@DeclareFontShape        {#1}{#2}{m}{scit}  {}%
  \Mn@DeclareFontShape        {#1}{#2}{m}{sscit} {}%
  \Mn@DeclareFontShape        {#1}{#2}{m}{sw}    {}%
  \Mn@DeclareFontShape        {#1}{#2}{sb}{sc}   {}%
  \Mn@DeclareFontShape        {#1}{#2}{sb}{ssc}  {}%
  \Mn@DeclareFontShape        {#1}{#2}{sb}{scit} {}%
  \Mn@DeclareFontShape        {#1}{#2}{sb}{sscit}{}%
  \Mn@DeclareFontShape        {#1}{#2}{sb}{sw}   {}%
  \Mn@DeclareFontShape        {#1}{#2}{b}{sc}    {}%
  \Mn@DeclareFontShape        {#1}{#2}{b}{ssc}   {}%
  \Mn@DeclareFontShape        {#1}{#2}{b}{scit}  {}%
  \Mn@DeclareFontShape        {#1}{#2}{b}{sscit} {}%
  \Mn@DeclareFontShape        {#1}{#2}{b}{sw}    {}%
}%
\otf@makeglobal{Mn@DeclareLargeFontFamily}

\newcommand*\Mn@DeclareSmallFontFamily[4]{%
  \DeclareFontFamily          {#1}{#2}{}%
  \Mn@DeclareFontShape        {#1}{#2}{m}{n}    {}%
  \Mn@DeclareFontShape        {#1}{#2}{m}{it}   {}%
  \Mn@DeclareFontShape        {#1}{#2}{sb}{n}   {}%
  \Mn@DeclareFontShape        {#1}{#2}{sb}{it}  {}%
  \Mn@DeclareFontShape        {#1}{#2}{b}{n}    {}%
  \Mn@DeclareFontShape        {#1}{#2}{b}{it}   {}%
  \Mn@DeclareFontSubstitutions{#1}{#2}%
}
\otf@makeglobal{Mn@DeclareSmallFontFamily}

\newcommand*\Mn@DeclareMathFontFamily[4]{%
  \DeclareFontFamily          {#1}{#2}{\skewchar\font=255}%
  \Mn@DeclareFontShape[math]  {#1}{#2}{m}{n}    {}%
  \Mn@DeclareFontShape[math]  {#1}{#2}{m}{it}   {}%
  \Mn@DeclareFontShape[math]  {#1}{#2}{m}{sc}   {}%
  \Mn@DeclareFontShape[math]  {#1}{#2}{m}{scit} {}%
  \Mn@DeclareFontShape[math]  {#1}{#2}{sb}{n}   {}%
  \Mn@DeclareFontShape[math]  {#1}{#2}{sb}{it}  {}%
  \Mn@DeclareFontShape[math]  {#1}{#2}{sb}{sc}  {}%
  \Mn@DeclareFontShape[math]  {#1}{#2}{sb}{scit}{}%
  \Mn@DeclareFontShape[math]  {#1}{#2}{b}{n}    {}%
  \Mn@DeclareFontShape[math]  {#1}{#2}{b}{it}   {}%
  \Mn@DeclareFontShape[math]  {#1}{#2}{b}{sc}   {}%
  \Mn@DeclareFontShape[math]  {#1}{#2}{b}{scit} {}%
  \Mn@DeclareFontSubstitutions{#1}{#2}%
}
\otf@makeglobal{Mn@DeclareMathFontFamily}

\newcommand*\Mn@DeclareFontSubstitutions[2]{%
  \DeclareFontShape           {#1}{#2}{bx}{n}  {<->ssub*#2/b/n}  {}%
  \DeclareFontShape           {#1}{#2}{bx}{it} {<->ssub*#2/b/it} {}%
  \DeclareFontShape           {#1}{#2}{m}{sl}  {<->ssub*#2/m/it} {}%
  \DeclareFontShape           {#1}{#2}{sb}{sl} {<->ssub*#2/sb/it}{}%
  \DeclareFontShape           {#1}{#2}{b}{sl}  {<->ssub*#2/b/it} {}%
  \DeclareFontShape           {#1}{#2}{bx}{sl} {<->ssub*#2/b/it} {}%
}
\otf@makeglobal{Mn@DeclareFontSubstitutions}

\ifx\@nodocument\relax
  \endgroup
\fi
%</fontdef>
%<*debug>
\iffalse
\renewcommand\DeclareFontFamily[3]{
  \edef\@tempa{\noexpand\DeclareFontFamily{#1}{#2}}%
  \@temptokena\expandafter{\@tempa{#3}}%
  \message{\the\@temptokena}%
}
\renewcommand\DeclareFontShape[6]{
  \edef\@tempa{\noexpand\DeclareFontShape{#1}{#2}{#3}{#4}{#5}}%
  \@temptokena\expandafter{\@tempa{#6}}%
  \message{\the\@temptokena}%
}
\fi
%</debug>
\endinput
%
%<Uextra>    \Mn@DeclareSmallFontFamily[Extra]{U}  {MinionPro}
%<LGR>       \Mn@DeclareSmallFontFamily       {LGR}{MinionPro}
%<LGI>       \Mn@DeclareSmallFontFamily       {LGI}{MinionPro}
%<OT1>       \Mn@DeclareLargeFontFamily       {OT1}{MinionPro}
%<T1>        \Mn@DeclareLargeFontFamily       {T1} {MinionPro}
%<LY1>       \Mn@DeclareLargeFontFamily       {LY1}{MinionPro}
%<T5>        \Mn@DeclareLargeFontFamily       {T5} {MinionPro}
%<T2A>       \Mn@DeclareSmallFontFamily       {T2A}{MinionPro}
%<T2B>       \Mn@DeclareSmallFontFamily       {T2B}{MinionPro}
%<T2C>       \Mn@DeclareSmallFontFamily       {T2C}{MinionPro}
%<TS1>       \Mn@DeclareLargeFontFamily       {TS1}{MinionPro}
%<X2>        \Mn@DeclareSmallFontFamily       {X2} {MinionPro}
%<OT2>       \Mn@DeclareSmallFontFamily       {OT2}{MinionPro}
%<OML&tosf>  \Mn@DeclareMathFontFamily        {OML}{MinionPro}
%<*OML&(lf|osf|tlf)>
  \@for\Mn@variant:=LF,TLF,OsF\do{%
    \DeclareFontFamily{OML}{MinionPro-\Mn@variant}{\skewchar\font=255}
    \@for\Mn@series:=m,sb,b,bx,eb\do{%
      \@for\Mn@shape:=n,it\do{%
        \DeclareFontShape{OML}{MinionPro-\Mn@variant}{\Mn@series}{\Mn@shape}%
          { <-> ssub*MinionPro-TOsF/\Mn@series/\Mn@shape }{}
      }%
    }%
  }%
%</OML&(lf|osf|tlf)>
%</fd>
%    \end{macrocode}
%
% \CheckSum{0}
% \CharacterTable
%  {Upper-case    \A\B\C\D\E\F\G\H\I\J\K\L\M\N\O\P\Q\R\S\T\U\V\W\X\Y\Z
%   Lower-case    \a\b\c\d\e\f\g\h\i\j\k\l\m\n\o\p\q\r\s\t\u\v\w\x\y\z
%   Digits        \0\1\2\3\4\5\6\7\8\9
%   Exclamation   \!     Double quote  \"     Hash (number) \#
%   Dollar        \$     Percent       \%     Ampersand     \&
%   Acute accent  \'     Left paren    \(     Right paren   \)
%   Asterisk      \*     Plus          \+     Comma         \,
%   Minus         \-     Point         \.     Solidus       \/
%   Colon         \:     Semicolon     \;     Less than     \<
%   Equals        \=     Greater than  \>     Question mark \?
%   Commercial at \@     Left bracket  \[     Backslash     \\
%   Right bracket \]     Circumflex    \^     Underscore    \_
%   Grave accent  \`     Left brace    \{     Vertical bar  \|
%   Right brace   \}     Tilde         \~}
%
% \Finale
\endinput
% vim:set ft=tex:

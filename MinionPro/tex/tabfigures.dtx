%\iffalse
%<*driver>
\NeedsTeXFormat{LaTeX2e}
\ProvidesFile{tabfigures.dtx}
%</driver>
%<package>\ProvidesPackage{tabfigures}
%<*driver|package>
[2007/03/31 v0.1b Tabular figures (patches)]
%</driver|package>
%<*driver>
\documentclass{ltxdoc}
\usepackage{MinionPro}
\usepackage{microtype}
\usepackage[toc,eqno,enum,bib,lineno]{tabfigures}
\usepackage{amsmath}
\usepackage{xcolor}
\usepackage{array,booktabs}
\usepackage{ifpdf}
\ifpdf\ifnum\pdftexversion<140 \else
\InputIfFileExists{glyphtounicode.tex}{\pdfgentounicode=1}{}
\fi\fi
% markup
\newcommand*\pkg[1]{\textsf{#1}}
\newcommand*\cls{} \let\cls\pkg
% show this package at work
\colorlet{tabular}{green!50!black}
\makeatletter
\def\Tf@font{\figureversion{tab}\color{tabular}}
\makeatother
\newenvironment{sample}{%
  \par
  \addvspace{\smallskipamount}%
  \minipage{\linewidth}%
}{%
  \endminipage
  \vspace{\smallskipamount}%
  \ignorespacesafterend
}
% metadata
\title{Tabular figures}
\author{Andreas B\"uhmann}
\date{\fileversion\ -- \filedate}
% hyperref
\usepackage[colorlinks=false,pdfborder={0 0 0},pdfusetitle]{hyperref}
\begin{document}
\DocInput{tabfigures.dtx}
\end{document}
%</driver>
%<*doc>
%\fi
%
% \GetFileInfo{tabfigures.dtx}
% \maketitle
%
% \section{Purpose}
% 
% Traditionally, there is no concept of tabular figures
% (\tabularfigures{0123456789}) versus proportional figures
% (\proportionalfigures{0123456789}) in \LaTeX. That is why, with a
% font like \pkg{MinionPro}, you have to specifically adapt your
% document class or even the \LaTeX\ kernel to this new task of using
% tabular figures in the right places such as the table of contents,
% enumerations or other material where numbers should line up
% vertically. This package tries to assist you in the most common
% situations.
%
% \section{Effects}
%
% This package is a collection of patches that inject
% |\figureversion{tabular}| into the involved commands in the right
% places. You have to explicitly enable those patches by using one or
% more of the following package options, like this:
% \begin{verbatim}
% \usepackage[toc,eqno]{tabfigures}
% \end{verbatim}
% This way, you can always turn the patches off and see for yourself
% if things break.  (While this package tries to be as robust as
% possible, it has to touch the internals of other classes and
% packages.)
%
% Throughout this document, tabular figures effected by this package
% are colored \textcolor{tabular}{green}.
%
% \begin{description}
% \item[\hyperdef{sample}{toc}{toc}] Use tabular figures for the
%   numbering and for the page numbers in the table of contents, list
%   of tables/figures, and similar lists.
%   \begin{sample}
%     \contentsline{section}{%
%       \numberline{4}Moving Information Around}{65}{sample.toc} %
%     \contentsline{subsection}{%
%       \numberline{4.1}The Table of Contents}{66}{sample.toc} %
%     \contentsline{subsection}{%
%       \numberline{4.2}Cross-References}{67}{sample.toc} %
%   \end{sample}
%
% \item[eqno] Use tabular figures in equation numbers.
%   \begin{sample}
%     \setlength{\abovedisplayskip}{0pt}%
%     \setlength{\belowdisplayskip}{0pt}%
%     \setcounter{equation}{18}%
%     \begin{align}
%       x&=y       & X&=Y       & a&=b+c\\
%       x'&=y'     & X'&=Y'     & a'&=b\\
%       x+x'&=y+y' & X+X'&=Y+Y' & a'b&=c'b
%     \end{align}
%   \end{sample}
% 
% \item[enum] Use tabular figures in enumerations.
%   \begin{sample}
%     \begin{enumerate} \setcounter{enumi}{38}
%     \item The world's fastest supercomputer will have its speed
%       measured in ``petaflops'', which represent 1,000 trillion
%       calculations per second.
%     \item The medical name for the part of the brain associated with
%       teenage sulking is ``superior temporal sulcus''.
%     \item Some Royal Mail stamps, which of course carry the Queen's
%       image, are printed in Holland.
%     \end{enumerate}
%   \end{sample}
% 
% \item[bib] Use tabular figures for the labels in the bibliography.
%   (The previous examples have been taken from the following sources.)
%   \begin{sample}
%     \renewcommand\section[2]{}%
%     \begin{thebibliography}{99}
%     \bibitem[19]{b19} Leslie Lamport. \LaTeX: A Document Preparation
%       System. Addison-Wesley, Reading, MA, 2nd Edition, 1994.
%     \bibitem[20]{b20} American Mathematical Society. User's Guide
%       for the \pkg{amsmath} Package (Version 2.0). Revised, 2002.
%     \bibitem[21]{b21} BBC News.
%       \href{http://www.bbc.co.uk/blogs/magazinemonitor/2006/12/100_things_we_didnt_know_last_2.shtml}%
%       {100 Things We Didn't Know Last Year.} 28 December 2006.
%     \end{thebibliography}
%   \end{sample}
%
% \item[lineno] Use tabular figures for line numbers. Only the
%   \pkg{doc} package is supported for now; other line-numbering
%   packages might be added later.
%   \begin{sample}
%     \setcounter{CodelineNo}{48}%
%    \begin{macrocode}
\DeclareOption{eqno}{%
  \AtBeginDocument{%
    \@ifpackageloaded{amsmath}{%
%    \end{macrocode}
% \end{sample}
% \end{description}
%
% \section{Compatibility}
%
% This package was designed to work with the default settings of the
% standard document classes \cls{article}, \cls{report}, and
% \cls{book}; their \pkg{KOMA-Script} counterparts; the \cls{memoir}
% class; and the \pkg{amsmath} package.
%
% \StopEventually
%
% \iffalse
%</doc>
%<*package>
% \fi
%
% \section{Implementation}
%
% \global\setcounter{CodelineNo}{0}
% Ease patching of commands. We store the original meaning of |\cmd|
% in a safe place and access it later (in the redefinition) with
% |\Tf@\cmd|.
%    \begin{macrocode}
\newcommand*\Tf@@[1]{Tf@@\expandafter\@cdr\string #1\@nil}
\newcommand*\Tf@[1]{\csname \Tf@@{#1}\endcsname}
\newcommand*\Tf@name{}
\newcommand*\Tf@save[1]{%
  \edef\Tf@name{\expandafter\expandafter\expandafter
    \noexpand\Tf@{#1}}%
  \expandafter\newcommand\Tf@name{}%
  \expandafter\let\Tf@name #1%
}
%    \end{macrocode}
% Just a shorthand.
%    \begin{macrocode}
\newcommand*\Tf@def[1]{%
  \Tf@save{#1}%
  \def#1%
}
%    \end{macrocode}
% Patch |\l@|\meta{level} commands. These commands always take two
% arguments, the second of which is the page number.
%    \begin{macrocode}
\newcommand*\Tf@pname{}
\newcommand*\Tf@patch@l[1]{%
  \@ifundefined{l@#1}{}{%
    \edef\Tf@pname{\expandafter\noexpand\csname l@#1\endcsname}%
    \expandafter\Tf@save\Tf@pname
    \expandafter\edef\Tf@pname##1##2{%
      \noexpand\Tf@
      \expandafter\noexpand\Tf@pname
      {##1}{\noexpand\Tf@font ##2}%
    }%
  }%
}
%    \end{macrocode}
% Debugging.
%    \begin{macrocode}
\newif\ifTf@debug \Tf@debugfalse
\DeclareOption{debug}{\Tf@debugtrue}
%    \end{macrocode}
%
% \subsection{Equation numbers}
% We distinguish between the two most frequent cases: \pkg{amsmath}
% and standard \LaTeX. All of the following patches work by injecting
% |\Tf@font| in the right place. They try to do this as robustly as
% possible by reusing the original definition.
%    \begin{macrocode}
\DeclareOption{eqno}{%
  \AtBeginDocument{%
    \@ifpackageloaded{amsmath}{%
      \Tf@def\maketag@@@#1{\Tf@\maketag@@@{\Tf@font #1}}%
    }{%
      \CheckCommand*\@eqnnum{{\normalfont \normalcolor (\theequation)}}%
      \def\@eqnnum{{\normalfont \Tf@font \normalcolor (\theequation)}}%
    }%
  }%
}
%    \end{macrocode}
% 
% \subsection{Table of contents}
% And similar lists such as list of figures and list of tables.
%    \begin{macrocode}
\DeclareOption{toc}{%
  \AtBeginDocument{%
%    \end{macrocode}
% Generic. First two command that are using in formatting the lists by
% default.
%    \begin{macrocode}
    \Tf@def\@dottedtocline#1#2#3#4#5{%
      \Tf@\@dottedtocline{#1}{#2}{#3}{#4}{\Tf@font #5}%
    }%
    \Tf@def\numberline#1{%
      \Tf@\numberline{\Tf@font #1}%
    }%
%    \end{macrocode}
% Then a bunch of |\l@|\meta{level} commands for usually available
% entry types, which might not use the commands from above.
%    \begin{macrocode}
    \Tf@patch@l{part}%
    \Tf@patch@l{chapter}%
    \Tf@patch@l{section}%
    \Tf@patch@l{subsection}%
    \Tf@patch@l{subsubsection}%
    \Tf@patch@l{paragraph}%
    \Tf@patch@l{subparagraph}%
    \Tf@patch@l{figure}%
    \Tf@patch@l{table}%
%    \end{macrocode}
% Special support for parts and chapters in \pkg{memoir}.
%    \begin{macrocode}
    \@ifclassloaded{memoir}{%
      \Tf@def\cftchapterpresnum{\Tf@font \Tf@\cftchapterpresnum}%
      \Tf@def\cftpartpresnum{\Tf@font \Tf@\cftpartpresnum}%
    }{}%
  }%
}
%    \end{macrocode}
% 
% \subsection{Enumerations}
% Labels in enumerations.
%    \begin{macrocode}
\DeclareOption{enum}{%
  \AtBeginDocument{%
    \Tf@def\labelenumi{\Tf@font \Tf@\labelenumi}%
    \Tf@def\labelenumii{\Tf@font \Tf@\labelenumii}%
    \Tf@def\labelenumiii{\Tf@font \Tf@\labelenumiii}%
    \Tf@def\labelenumiv{\Tf@font \Tf@\labelenumiv}%
  }%
}
%    \end{macrocode}
% 
% \subsection{Bibliography}
% Labels in the bibliography.
%    \begin{macrocode}
\DeclareOption{bib}{%
  \AtBeginDocument{%
    \Tf@def\@biblabel{\Tf@font \Tf@\@biblabel}%
  }%
}
%    \end{macrocode}
%
% \subsection{Line numbers}
%    \begin{macrocode}
\DeclareOption{lineno}{%
  \AtBeginDocument{%
    \@ifpackageloaded{doc}{%
      \CheckCommand*\theCodelineNo{%
        \reset@font\scriptsize\arabic{CodelineNo}}%
      \def\theCodelineNo{%
        \reset@font\Tf@font\scriptsize\arabic{CodelineNo}}%
    }{}%
  }%
}
\ProcessOptions\relax
%    \end{macrocode}
%
% \subsection{Auxiliary macros}
% This command is used for switching to tabular figures. This can be
% redefined to allow debugging, disabling, etc.
%    \begin{macrocode}
\newcommand*\Tf@font{\figureversion{tabular}}
%    \end{macrocode}
% Visual debugging: Set tabular figures (produced by this package) in
% green.
%    \begin{macrocode}
\ifTf@debug
  \RequirePackage{xcolor}%
  \colorlet{Tf@debug}{green!50!black}%
  \renewcommand\Tf@font{\figureversion{tabular}\color{Tf@debug}}%
\fi
%    \end{macrocode}
% Check if figure versions are supported at all. If not, we cannot do
% anything useful.
%    \begin{macrocode}
\AtBeginDocument{%
  \@ifundefined{figureversion}{%
    \PackageWarning{tabfigures}{There is no
      \string\figureversion\space to support tabular figures}%
    \let\Tf@font\@empty
  }{}%
}
%    \end{macrocode}
% That's it for now.
% \endinput
%</package>
\endinput
